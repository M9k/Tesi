\documentclass[Tesi.tex]{subfiles}

\begin{document}
\setcounter{chapter}{0}
\chapter{Contesto aziendale}

Nata nel 1987, Wintech Comumnication Factory SPA è ad oggi uno dei pochi \gloss{System Integrator} capace di vantare una lunga tradizione e un patrimonio di conoscenze nei diversi ambiti di competenza del settori ICT. \\
Tra le partnership si possono citare IBM, Symantec, Hewlett Packard, Microsoft, Oracle, VMware e Sophos.
\begin{figure}[H]
	\centering
	\includegraphics[width=0.7\linewidth]{"images/LogoWintech"}
	\caption{Logo di Wintech}
	\label{fig:Logo di Wintech}
\end{figure}

\section{Dominio applicativo}
Wintech opera in un dominio molto vasto, che spazia dal Cloud alla \gloss{Digital Transformation}, per tale motivo in questa sezione verrà analizzato unicamente il dominio relativo allo stage conseguito, cioè quello del monitoraggio e della sicurezza fisica delle reti. \\

\subsection{Com'è cambiata la Cybersecurity negli ultimi anni?}
Il panorama attuale per tipologia di attacchi e motivazioni, è profondamente mutato rispetto anche solo a qualche anno fa. Tutti gli osservatori sulla tematica Cybersecurity lo confermano con i dati. Non è fare terrorismo psicologico, ma essere realisti, quando si afferma che tutti si è potenzialmente vulnerabili e bersagli del cybercrime e quindi "is not a matter of if, but a matter of when".\\
Bisogna sfatare l'immaginario collettivo in cui siano solo le grandi società americane, grandi brand, ad essere attaccate per attivismo. Le motivazioni sono notevolmente mutate e hanno come target qualsiasi azienda, anche le realtà della piccole e medie imprese che costituiscono il tessuto delle aziende italiane sono pesantemente bersagliate.

\subsection{La proposta di Wintech}
La proposta per consentire di mantenere un elevato livello di sicurezza all'interno delle proprie reti proposto da Wintech si compone di svariate tecnologie, che vanno ad integrarsi per permettere una protezione completa su tutti i possibili fronti di attacco. \\
Le tecnologie proposte saranno riguardanti il monitoraggio della rete, il controllo della configurazione degli apparati ed il port-based Network Access Control.

\section{Tipologia di clientela aziendale}
%TODO mettere o no?
TODO mettere o no?

\section{Processi aziendali}
%TODO chiedere a Lisa
TODO chiedere a Lisa

\end{document}
