\documentclass[Tesi.tex]{subfiles}
\begin{document}
\clearpage\thispagestyle{empty}
	
\renewcommand{\chaptername}{}
\renewcommand{\thechapter}{}
\chapter{Sommario}
Questo documento si prefigge lo scopo di presentare il lavoro svolto durante lo stage presso Wintech Communications Factory. \\
L'attività è stata di natura sistemistica, andando a monitorare e a migliorare la sicurezza di una Campus Area Network di elevata complessità, composta da oltre 90 switch di rete. \\\\

Con l'avanzamento delle nuove tecnologie succede sempre più di frequente che le aziende debbano estendere la propria rete, incrementandone la complessità e quindi le possibili problematiche, le quali devono essere correttamente controllate e gestite. \\\\
L'attività svolta ha riguardo il controllo in tempo reale dello stato di una rete di dimensioni considerevoli. L'obiettivo è l'individuazione dei colli di bottiglia e delle problematiche ai dispositivi, sia hardware che software, al fine di permettere interventi tempestivi o preventivi.\\
L'attenzione è stata posta anche sul disaster recovery in caso di guasti, salvando periodicamente tutte le configurazioni in uso dai dispositivi, in modo tale da permettere una sostituzione rapida degli apparati. \\
Ulteriormente è stata trattata la sicurezza fisica delle reti, impedendo un accesso non autorizzato a risorse di elevata criticità mediante connessione Wi-Fi a coloro che non possiedono dei privilegi di livello adeguato. \\


\newpage
\chapter{Convenzioni tipografiche}
Per favorire la lettura del documento e la sua comprensione è stato introdotto un glossario. \\
Qualora un termine presente nel glossario comparisse all'interno del testo, in un contesto nel quale si suppone il lettore possa non comprenderlo, verrà evidenziato rispetto al paragrafo presentandolo scritto in corsivo e apponendogli in conclusione una "G" al pedice. \\\\
Nel caso si stesse visionando la versione digitale di questo documento è possibile essere reindirizzati direttamente al termine nel glossario semplicemente cliccandoci sopra. \\


\clearpage
\end{document}
