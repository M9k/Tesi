\documentclass[Realizzazione.tex]{subfiles}

\begin{document}
\section{Implementazione}

\begin{itemize}
	\item \textbf{Periodo previsto}: dal 25/06/2018 al 06/07/2018;
	\item \textbf{Numero di ore previste}: 80h;
	% TODO completare ore
	\item \textbf{Periodo effettivo}: ;
	\item \textbf{Numero di ore effettive}: .
\end{itemize}

%TODO
\subsection{Laboratorio per lo sviluppo}
Prima di procedere alla produzione si è ovviamente testato le configurazioni in un laboratorio. \\
Per tale fine sono stati utilizzati uno switch HP 2530-8G PoE+ e svariati access point MikroTik cAP ac RBcAPGi-5acD2nD, collegati a un router MikroTik CCR1009-7G-1C-1S+.

\begin{figure}[H]
	\centering
	\includegraphics[width=0.6\linewidth]{"images/HP"}
	\caption{HP 2530-8G PoE+}
	\label{fig:HP 2530-8G PoE+}
\end{figure}
\begin{figure}[H]
	\centering
	\includegraphics[width=0.6\linewidth]{"images/AP"}
	\caption{MikroTik cAP ac}
	\label{fig:MikroTik cAP ac}
\end{figure}
\begin{figure}[H]
	\centering
	\includegraphics[width=0.6\linewidth]{"images/Router"}
	\caption{Router MikroTik}
	\label{fig:Router MikroTik}
\end{figure}

\subsection{Inserimento su DNS interno degli hosts} 
Per consentire una adeguata imlementazione di Observium e rConfig era richiesto l'utilizzo dei nomi DNS degli apparati e non dell'indirizzo IP. \\
Si è dunque proceduto a comunicare la lista dei nomi e dei relativi IP dei dispositivi al responsabile affinché vengano inseriti nel sistema, in quanto non c'era la disponibilità di accedere direttamente alle impostazioni del DNS.

\subsection{Implementazione Virtual Machine}
Per consentire l'esecuzione continua dei software sono state create due Virtual Machine, o macchine virtuali.\\
Il server utilizzato per tale fine è quello del cliente, utilizzante le tecnologie prodotte da VMware per la virtualizzazione. \\
Le due macchine virtuali sono state entrambe create con Linux, in quanto gratuito e già conosciuto nel contesto aziendale.

\subsection{Configurazione di base software RConfig}
Per la configurazione di rConfig si è proceduto ad automatizzare l'inserimento dei dati, che altrimenti avrebbe consumato una ingente quantità di tempo visto il numero di dispositivi presenti. \\
Come base di riferimento per l'inserimento dei dati si è attinto dalla tabella dei dispositivi 
precedentemente realizzata, che è stata esportata in formato \gloss{CSV}.\\
Successivamente si è proceduto a modificare il formato dei dati esportati, operando per mezzo di una \gloss{espressione regolare}, al fine di convertirlo in una matrice in linguaggio JavaScript per il suo utilizzo all'interno del browser. \\
Si è proceduto con lo sviluppo di uno script JavaScript, visionabile in \hyperref[sec:Appendice B]{Appendice B}, che mediante il plug-in GreaseMonkey consentiva il caricamento automatico dei dati precedentemente posti in forma corretta. \\
Per terminare è stato eseguito lo script, che ha proceduto all'inserimento degli apparati. \\

\subsection{Configurazione di base software monitoraggio Observium}
Analogamente con quanto effettuato con RConfig si è proceduto all'inserimento di tutti i dispositivi e alla loro configurazione in Observium. \\
Questa operazione è stata più complessa rispetto all'altro software, in quanto richiedeva una aggiunta iniziale di ogni dispositivo, seguito dall'attesa della sua identificazioni per poi terminare con l'aggiunta delle informazioni aggiuntive. Seguendo quanto deciso nella fase di progettazione si è andato ad inserire, sempre mediante l'aiuto di GreaseMonkey, la sua locazione e le sue coordinate GPS. \\
Per completare l'attività si è anche dovuto segnalare al software tutte le interfacce non utilizzate, in modo tale che non venissero generati avvisi a causa della loro inoperatività.\\

Al termine dell'operazione gli elementi inseriti all'interno del software erano i seguenti:
\begin{itemize}
	\item 87 switch;
	\item 6 apparati wireless;
	\item 2923 interfacce;
	\item 395 sensori;
	\item 269 periferiche.
\end{itemize}

I sensori presenti sono principalmente relativi all'alimentazione ed in alcuni casi alla temperatura del dispositivo, mentre i componenti aggiuntivi sono ventole di raffreddamento ed alimentazione. \\
Questi numeri dimostrano l'impossibilità di mantenere un controllo soddisfacente della rete utilizzando la versione gratuita di PRTG Netowrk Monitor, che consentiva in totale il monitoraggio di 1000 elementi. \\

\subsection{Predisporre servizio NPS ( Radius ) sui due Active Directory servers} 
%TODO
asd
\subsection{Creazione della configurazione di test per laboratorio con nuove funzionalità di sicurezza} 
%TODO
asd
\subsection{Test nuove funzionalità con switch laboratorio} 
%TODO
asd
\subsection{Creazione nuove configurazioni per tutti gli switch del Campus} 
%TODO
asd
\subsection{Caricamento configurazioni negli switch del Campus} 
%TODO
asd
\subsection{Test di base nuove funzionalità di sicurezza implementate} 
%TODO
asd

\subsection{Scheduling backup delle impostazioni con rConfig}

Per completare la messa in funzione di rConfig si è proceduto alla suddivisione dei dispositivi in gruppi secondo la loro locazione ed alla predisposizione di backup periodici. \\
Per non sovraccaricare la rete si è scelto di svolgere il backup a cadenza settimanale durante la notte, facendo attenzione a non sovrapporsi al backup giornaliero degli altri apparati. Gli orari stati scelti in modo tale che i task non si sovrappongano tra di loro, in modo di evitare eventuali problemi.\\

Questa attività si è rilevata più ostica del previsto a causa di molte problematiche di rConfig, rimaste per ora irrisolte anche nella versione in sviluppo. \\
Inizialmente si sono notati un numero di backup superiori a quanto schedulato. Questo avveniva perché le attività inserite e poi rimosse non risultavano più presenti dall'interfaccia web, ma rimanevano in esecuzione. Per arginare questo problema si è dovuto accedere direttamente al database dell'applicazione mediante MySql e correggere le informazioni in esso contenute. \\
Durante la correzione del problema precedente si sono rilevate altre anomalie all'interno del database, soprattutto dovute a una scarsa normalizzazione dei dati e a una strutturazione non adatta ad un database di tipo SQL. Pertanto si è dovuto controllare eventuali incongruenze e correggerle, in modo da evitare comportamenti inaspettati in futuro. \\


\subsection{Aggiornamento della documentazione progettuale}
Durante questa fase si è proceduto all'aggiornamento della documentazione, andando a descrivere le funzionalità dei programmi e delle tecnologie utilizzate, il loro utilizzo e tutti i problemi riscontrati, correlati dalla soluzione applicata per correggerli.
	
\end{document}
