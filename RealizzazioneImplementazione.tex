\documentclass[Realizzazione.tex]{subfiles}

\begin{document}
\section{Implementazione}

\begin{itemize}
	\item \textbf{Periodo previsto}: dal 25/06/2018 al 06/07/2018;
	\item \textbf{Numero di ore previste}: 80h;
	\item \textbf{Periodo effettivo}: dal 27/06/2018 al 11/07/2018;
	\item \textbf{Numero di ore effettive}: 88h.
\end{itemize}

L'implementazione di quanto progettato ha richiesto un numero di ore maggiore di quanto preventivato a causa delle problematiche sorte dall'utilizzo di rConfig. Una parte di queste ora aggiuntive sono però state ammortizzate dal risparmio di tempo avuto nella fase di data-entry relativa ai dispositivi, in quanto è stato ampliamene velocizzato dall'utilizzo di scripting.

\subsection{Preparazione dell'ambiente}

\subsubsection{Inserimento dei dispositivi sul DNS interno} 
Per consentire una adeguata imlementazione di Observium e rConfig era richiesto l'utilizzo dei nomi DNS degli apparati e non dell'indirizzo IP. \\
Si è dunque proceduto a comunicare la lista dei nomi e dei relativi IP dei dispositivi al responsabile affinché vengano inseriti nel sistema, in quanto non c'era la disponibilità di accedere direttamente alle impostazioni del DNS.

\subsubsection{Implementazione Virtual Machine}
Per consentire l'esecuzione continua dei software sono state create due Virtual Machine, o macchine virtuali.\\
Il server utilizzato per tale fine è quello del cliente, utilizzante le tecnologie prodotte da VMware per la virtualizzazione. \\
Le due macchine virtuali sono state entrambe create con CentOS Linux, in quanto gratuito e già conosciuto nel contesto aziendale.

\subsubsection{Creazione laboratorio per le nuove politiche}
Prima di procedere alla produzione si sono testate le configurazioni in un laboratorio. \\
Per tale fine sono stati utilizzati uno switch HP 2530-8G PoE+ e svariati Access Point MikroTik cAP ac RBcAPGi-5acD2nD, collegati a un controller MikroTik CCR1009-7G-1C-1S+.

\begin{figure}[H]
	\centering
	\includegraphics[width=0.6\linewidth]{"images/HP"}
	\caption{Switch HP 2530-8G PoE+}
	\label{fig:Switch HP 2530-8G PoE+}
\end{figure}
\begin{figure}[H]
	\centering
	\includegraphics[width=0.6\linewidth]{"images/AP"}
	\caption{Access Point MikroTik cAP ac}
	\label{fig:Access Point MikroTik cAP ac}
\end{figure}
\begin{figure}[H]
	\centering
	\includegraphics[width=0.6\linewidth]{"images/Router"}
	\caption{Controller MikroTik}
	\label{fig:Controller MikroTik}
\end{figure}

\subsection{Configurazione del monitoraggio con Observium}
Analogamente con quanto effettuato con RConfig è stato effettuato l'inserimento di tutti i dispositivi e alla loro configurazione in Observium. \\
Questa operazione è stata più complessa rispetto all'altro software, in quanto richiedeva una aggiunta iniziale di ogni dispositivo, seguito dall'attesa della sua identificazioni per poi terminare con l'aggiunta delle informazioni aggiuntive. Seguendo quanto deciso nella fase di progettazione si è andato ad inserire, sempre mediante l'aiuto di GreaseMonkey, la sua locazione e le sue coordinate GPS. \\\\
Per completare l'attività si è anche dovuto segnalare al software tutte le interfacce non utilizzate, in modo tale che non venissero generati avvisi a causa della loro inoperatività.\\

Al termine dell'operazione gli elementi inseriti all'interno del software erano i seguenti:
\begin{itemize}
	\item 94 apparati di rete;
	\item 7 apparati wireless;
	\item 3064 interfacce;
	\item 452 sensori;
	\item 272 periferiche.
\end{itemize}

I sensori presenti sono principalmente relativi all'alimentazione ed in alcuni casi alla temperatura del dispositivo, mentre i componenti aggiuntivi sono ventole di raffreddamento ed alimentazione. \\\\
Questi numeri dimostrano l'impossibilità di mantenere un controllo soddisfacente della rete utilizzando la versione gratuita di PRTG Netowrk Monitor, che consentiva in totale il monitoraggio di 1000 elementi.

\subsection{Configurazione del versioning con RConfig}
Per la configurazione di rConfig si è proceduto ad automatizzare l'inserimento dei dati, che altrimenti avrebbe consumato una ingente quantità di tempo visto il numero di dispositivi presenti. \\\\
Come base di riferimento per l'inserimento dei dati si è attinto dalla tabella dei dispositivi precedentemente realizzata, che è stata esportata in formato \gloss{CSV}.\\
Successivamente si è modificato il formato dei dati esportati, operando per mezzo di una \gloss{espressione regolare}, al fine di convertirlo in una matrice in linguaggio JavaScript per il suo utilizzo all'interno del browser. \\
Si è quindi sviluppato uno script JavaScript, visionabile in \hyperref[sec:Appendice B]{Appendice B}, che mediante il plug-in GreaseMonkey consentiva il caricamento automatico dei dati precedentemente posti in forma corretta. \\
Per terminare è stato eseguito lo script, che ha proceduto all'inserimento degli apparati. \\


\subsubsection{Scheduling backup delle impostazioni con rConfig}

Per completare la messa in funzione di rConfig si sono suddivisi i dispositivi in gruppi secondo la loro locazione ed si sono predisposti dei backup periodici. \\
Per non sovraccaricare la rete si è scelto di svolgere il backup a cadenza settimanale durante la notte, facendo attenzione a non sovrapporsi al backup giornaliero degli altri apparati. Gli orari sono stati scelti in modo tale che i task non si sovrappongano tra di loro, così da evitare eventuali problemi.\\

Questa attività si è rilevata più ostica del previsto a causa di molte problematiche di rConfig, rimaste per ora irrisolte anche nella versione più recente. \\
Inizialmente si sono notati un numero di backup superiori a quanto schedulato. Questo avveniva perché le attività inserite e poi rimosse non risultavano più presenti dall'interfaccia web, ma rimanevano in esecuzione. Per arginare questo problema si è dovuto accedere direttamente al database dell'applicazione mediante MySql e correggere le informazioni in esso contenute. \\\\
Durante la correzione del problema precedente si sono rilevate altre anomalie all'interno del database, soprattutto dovute a una scarsa normalizzazione dei dati e a una strutturazione non adatta ad un database di tipo SQL. Pertanto si è dovuto controllare eventuali incongruenze e correggerle, in modo da evitare comportamenti inaspettati in futuro. \\\\
Nell'immagine seguente si può notare un esempio di quanto affermato. I dispositivi, chiamati nodi, non hanno una relazione molti a molti con i task, ma bensì contengono delle colonne con il nome riferito all'id del tast, il cui valore è un enumeratore che ne indica l'abilitazione.
\begin{figure}[H]
	\centering
	\includegraphics[width=0.6\linewidth]{"images/rconfig_normalizzazione"}
	\caption{Esempio della scarsa normalizzazione dei dati in rConfig}
	\label{fig:Esempio della scarsa normalizzazione dei dati in rConfig}
\end{figure}

\subsubsection{Backup configurazioni MikroTik}
Una volta avuto a disposizione i device MikroTik è emerso un problema inaspettato con rConfig, che non riusciva a completare il backup delle configurazioni su tali dispositivi. \\
Il problema era causato da una serie di caratteri non stampati a schermo utilizzati dai dispositivi, che rConfig non sapeva rilevare correttamente, impedendogli di compiere i passaggi di login necessari. \\
Per risolvere tale problema in modo veloce, efficiente e duraturo nel tempo si è scelto di incaricare il dispositivo stesso dell'effettuazione del backup. Per raggiungere tale scopo è stato creato uno script, visibile in \hyperref[sec:Appendice C]{Appendice C}, che permette il salvataggio automatico delle proprie configurazioni su un server FTP. \\
Una volta verificato il corretto funzionamento dello script è stato schedulato in modo tale avvenga automaticamente ad intervalli prefissati. \\\\
Ulteriormente il server FTP è stato dotato di un demone HTTP Apache, in modo da consentire la navigazione dei backup effettuati. Per migliorare la fruizione del sito si è utilizzato Directory Lister, il quale è stato modificato per richiedere l'inserimento di una password. \\

\begin{figure}[H]
	\centering
	\includegraphics[width=1\linewidth]{"images/DirectoryLister"}
	\caption{Lista delle cartelle con i backup in Directory Lister}
	\label{fig:Lista delle cartelle con i backup in Directory Lister}
\end{figure}

\newpage
\subsection{Configurazione nuovi apparati di rete}
Per il collegamento tra gli access point e il router si è andati ad utilizzare una VLAN ad essi dedicata. \\
Sopra di essa è stato creato un datapath costituito da un bridge che consenta la comunicazione dei dispositivi connessi alle reti wifi. \\
Per motivi di sicurezza si è scelto di creare tre tunnel Ethernet over IP (EoIP) per incanalare i tre flussi di dati provenienti dai tre SSID wifi. \\
Così facendo si è ottenuta una elevata sicurezza, unita alla facilità di installazione di un nuovo dispositivo, in quanto è necessario solamente trasportargli la VLAN che utilizza, nell'immagine sottostante denominata "VLAN AP", e non quelle presenti nel tunnel, denominato "BRIDGE DATA". \\\\
Lo svantaggio di questa modalità è l'aver aggiunto un piccolo overhead ai pacchetti, che risulta trascurabile a fronte della semplicità di estensione e della elevata sicurezza che offre.

\begin{figure}[H]
	\centering
	\includegraphics[width=0.6\linewidth]{"images/VLAN"}
	\caption{Configurazione della rete}
	\label{fig:Configurazione della rete}
\end{figure}

\subsection{Attivazione servizio NPS in Active Directory} 
Per consentire l'autenticazione tramite le credenziali Active Directory è stato predisposto il servizio NPS, in italiano denominato "criteri di rete". \\
Innanzitutto è stato abilitato il servizio NPS, per poi essere successivamente configurato. Si è poi proceduto con l'abilitazione del 802.1X, aggiungendo i clients RADIUS, che nel nostro caso è unicamente il MikroTik Cloud Core Router.\\
Prima di definire le politiche è stato creato un gruppo di utenti, il quale conterrà tutti coloro che avranno il privilegio di usufruire delle rete CORPORATE e MOBILE. \\\\
Le politiche di accettazione delle richieste possono essere riepilogate nel seguente modo:
\begin{enumerate}
\item Accetta se l'indirizzo IP appartiene alla rete Corporate, viene fornito il certificato corretto e le credenziali appartengono ad un utente inserito nel relativo gruppo;
\item Accetta se l'indirizzo IP appartiene alla rete Mobile e le credenziali appartengono ad un utente inserito nel relativo gruppo;
\item Rifiuta.
\end{enumerate}

\begin{figure}[H]
	\centering
	\includegraphics[width=1\linewidth]{"images/nps_conf"}
	\caption{Politiche di sicurezza utilizzate}
	\label{fig:Politiche di sicurezza utilizzate}
\end{figure}


Successivamente le informazioni di accesso ai servizi NPS sono state inserite nella configurazione del Router MikroTik, in modo da permetterne la comunicazione tramite protocollo RADIUS.

\begin{figure}[H]
	\centering
	\includegraphics[width=1\linewidth]{"images/nps_miktorik"}
	\caption{Configurazione RADIUS del controller MikroTIk}
	\label{fig:Configurazione RADIUS del controller MikroTIk}
\end{figure}

\newpage
\subsection{Creazione del Captive Portal}
La creazione del Captive Portal è stata effettuata su una macchina virtuale nuova, in modo tale da garantirne una segregazione nel sistema e delle regole nel firewall specifiche alla sua funzionalità. \\
Innanzitutto è stata effettuata l'installazione del sistema operativo CentOS 7, che è stato scelto perché già conosciuto ed ampiamente utilizzato nell'ambito aziendale.
Successivamente si sono andati ad installare tutti i servizi, o "demoni" nello slang UNIX, in modo da offrire tutte le funzionalità richieste, brevemente riportati nella tabella sottostante.

\label{table:Servizi per l'implementazione del Captive Portal}
\rowcolors{2}{CRighePari}{CRigheDispari}
\renewcommand*{\arraystretch}{1.2}
\begin{longtable}[H]{p{2.6cm}|p{6cm}|p{6cm}}
	\rowcolor{CHeader}
	\color{CHeaderText} \textbf{Demone} & \color{CHeaderText} \textbf{Funzionalità} & \color{CHeaderText} \textbf{Utilizzo} \\
	\endhead
		HttpD
			& Servizio di server web, dedito alla gestione delle richieste http, risoluzione del codice PHP mediante apposito modulo ed invio di pagine web.
			&  Gestione ed interazione dell'interfaccia del Captive Portal. \\
		MySqlD
			& Servizio database, dedito al mantenimento e alla organizzazione dei dati, assieme al controllo d'accesso.
			& Gestione delle informazioni relative alla registrazione degli utenti, agli account abilitati e raccoglitore di informazioni per RadD. \\
		CronD
			& Servizio scheduler, dedito alla pianificazione dell'esecuzione dei comandi.
			& Controllo periodico degli account scaduti, che dovranno essere rimossi. \\
		RadD
			& Servizio RADIUS, accede a una serie di direttive legate agli accessi e risponde in modo affermativo o negativo a delle richieste di autenticazione.
			& Utilizzato per determinare la validità degli account con i quali si effettua l'accesso nella rete. \\
		FirewallD
			& Servizio firewall preinstallato in CentOS, determina quali connessioni sono permesse e quali invece devono essere bloccate.
			& Utilizzato per impedire un accesso dall'esterno ai servizi che non necessitano una iterazione dalla rete, ma solamente in locale, ad esempio MySqlD. \\
		Sendmail
			& Servizio email, dedito al trasferimento di messaggi mediante SMTP.
			& Consente di inviare l'e-mail con la richiesta di approvazione allo sponsor e, in caso di accettazione, avvisa l'utente comunicandogli le informazioni necessarie all'accesso. \\
	\hiderowcolors
	\caption{Servizi per l'implementazione del Captive Portal}
\end{longtable}


\subsubsection{Sperimentazione Captive Portal sicuro}
Durante la realizzazione del Captive Portal è stato deciso di effettuare un rapido test riguardante l'utilizzo di HTTPS con un certificato self-signed, in modo tale da impedire attacchi \gloss{Man In The Middle} durante l'autenticazione. \\\\
Purtroppo la modalità di redirezione utilizzata dagli hotspot è esplicitamente vietata nel protocollo HTTPS, però la sua implementazione varia secondo il produttore del sistema operativo. \\
Si è analizzato il comportamento su due dispositivi, un Huawei P8 con ROM EMUI e un Xiaomi Redmi 4X con ROM MIUI China Dev. EMUI, una volta connessa alla rete, rifiutava la risposta in HTTPS, in quanto violava lo standard, non rilevando quindi la presenza del captive portal ed impedendo all'utente di usufruire della navigazione. MIUI invece si comportava in modo più permissivo, ignorando l'infrazione del protocollo e presentando il login all'utente.\\\\
A causa di tali problematiche non si è proceduto all'abilitazione di HTTPS nel Captive Portal, mantenendo quindi la connessione in HTTP.

\subsection{Inserimento regole Firewall} 
Una volta terminata la configurazione della rete è stato necessario, prima del suo utilizzo, inserire nuove regole all'interno del firewall per consentire il soddisfacimento dei requisiti relativi alle politiche di sicurezza precedentemente presentati. \\
Tali regole potevano essere inserite anche nell'apparato Router di MikroTik, ma considerando la sua dotazione hardware si è preferito delegare il compito a dispositivi specifici, già presenti nel sistema. \\
Per eseguire tale operazione ho dovuto preparare la lista delle regole necessarie e comunicarle a chi di dovere affinché venissero applicate, in quanto mi era precluso l'accesso al firewall.

\subsection{Test delle configurazioni in laboratorio} 
Prima di inserire il sistema in produzione si è andato a testare il funzionamento e la sua estensibilità. \\
Innanzitutto si è proceduto ad aumentare il numero di Access Point connessi in contemporanea, passando da 1 a 4, in modo tale da verificare la complessità della configurazione iniziale. \\
Per ogni nuovo dispositivo è stato sufficiente abilitare CAPsMAN sulle interfacce wireless affinché divenga operativo con tutte le misure di sicurezza abilitate. \\
Si è anche definita una password di accesso alla configurazione, che comunque non sarebbe stata raggiungibile da parte degli utilizzatori della connettività grazie al tunnel EoIP. \\
L'accesso ad internet si è rivelato funzionante e tutti i dispositivi venivano inseriti all'interno della  VLAN con le corrispettive configurazioni fornite dal DHCP.

\subsubsection{Test delle funzionalità di sicurezza in laboratorio} 
Un altro aspetto di elevata importanza è stata la sicurezza fornita dal nuovo sistema. \\
Innanzitutto si è verificata la stabilità del captive portal, cercando di forzare il login mediante attacchi \gloss{SQL injection} utilizzando SQLmap e VEGA. \\
I risultati ottenuti sono stati soddisfacenti, in quanto un preciso controllo dei valori inseriti all'interno dei campi di testo impediva di aggirare l'autenticazione.
\begin{figure}[H]
	\centering
	\includegraphics[width=0.9\linewidth]{"images/vega"}
	\caption{Risultato ottenuto dalla analisi di VEGA}
	\label{fig:Risultato ottenuto dalla analisi di VEGA}
\end{figure}

Successivamente si è andata a verificare la protezione ad eventuali attacchi eseguiti una volta già autenticati, come ad esempio il \gloss{Man In The Middle}. \\
Non sono state rilevate vulnerabilità, in quanto gli altri dispositivi presenti nella rete risultavano irraggiungibili, e di conseguenza anche software come Wireshark e Arcai's NetCut non riuscivano a rilevare gli altri utenti. \\
Una vulnerabilità implicita nel protocollo WiFi, impossibile da risolvere per la sua natura, è la cattura in monitor mode dei pacchetti che transitano sulla rete, permettendo quindi di identificare i dispositivi che usufruiscono della rete. Però anche questa vulnerabilità implicita risulta di scarsa utilità per un attaccante, in quanto la configurazione degli apparati non gli permettono di accedere agli altri dispositivi e manometterli. \\\\
Nella immagine sottostante si può notare a sinistra una analisi effettuata in una rete del cliente prima della installazione del nuovo sistema, con rilevati 29 dispositivi connessi, mentre a destra l'analisi effettuata nella nuova configurazione, senza nessun utente connesso visibile.
\begin{figure}[H]
	\centering
	\includegraphics[width=1\linewidth]{"images/netcut"}
	\caption{Risultati ottenuti con NetCut}
	\label{fig:Risultati ottenuti con NetCut}
\end{figure}


\subsection{Aggiornamento della documentazione progettuale}
Durante questa fase è stata aggiornata la documentazione progettuale, andando a descrivere le funzionalità dei programmi e delle tecnologie utilizzate, il loro utilizzo e tutti i problemi riscontrati, correlati dalla soluzione applicata per correggerli.
	
\end{document}
