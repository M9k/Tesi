\documentclass[Tesi.tex]{subfiles}

\begin{document}
\chapter{Tecnologie}

Questa sezione illustra le principali tecnologie incontrate ed utilizzate nel corso dello stage per il controllo, il monitoraggio e la sicurezza delle reti di loro competenza. \\\\
Essendo la sicurezza informatica un settore nel quale gli standard, le leggi e le tipologie di attacco evolvono molto in fretta, le tecnologie utilizzate devono adeguarsi di conseguenza. Questo porta i tool ed i programmi utilizzati a diventare velocemente obsoleti oppure a variare con il passare del tempo.

\section{Tecnologie per il monitoraggio}
\subsection{PRTG Network Monitor}
PRTG è un sistema di monitoraggio della rete utilizzato nella rete analizzata durante lo stage. Il suo fine è la rilevazione delle anomalie, ma a causa delle limitazioni della versione gratuita e del suo elevato costo si è scelto di sostituirlo.\\
Il software consiste in un servizio al quale ci si collega mediante il client fornito oppure attraverso una interfaccia web e permette di tenere sotto controllo anche siti web e servizi di varia natura.\\\\
Questo software era già presente all'interno del sistema ed è stata analizzato durante la prima fase dello stage.
\begin{figure}[H]
	\centering
	\includegraphics[width=0.35\linewidth]{"images/logo/PRTG_logo"}
	\caption{Logo di PRTG}
	\label{fig:Logo di PRTG}
\end{figure}

\subsection{Observium}
Observium è un sistema di monitoraggio della rete compatibile con i dispositivi delle principali aziende produttrici di apparati di rete. \\
La sua implementazione durante l'attività formativa è stata effettuata per sopperire ai limiti posti da PRTG e migliorare quindi la qualità dei servizi. \\\\
Tra i dispositivi supportati si possono citare Cisco, Dell, HP, Huawei, Lenovo, MikroTik, Netgear e ZTE. \\
Il software dispone anche di una ricerca automatica dei dispositivi presenti all'interno della rete, funzionalità utile per reti di piccola-media dimensione. \\\\
Le funzionalità che fornisce sono il monitoraggio del \gloss{QOS}, il raggruppamento dei dispositivi mediante regole definite dall'utente e l'alert automatico nel caso vengano superate determinate soglie.\\\\
Questa tecnologia è stata implementata durante lo svolgimento dell'attività di stage.

\begin{figure}[H]
	\centering
	\includegraphics[width=0.4\linewidth]{"images/logo/Observium_logo"}
	\caption{Logo di Observium}
	\label{fig:Logo di Observium}
\end{figure}

\subsection{RConfig}
Il tool RConfig è un configuration management mirato ai dispositivi di rete che permette in modo veloce ed automatizzato di effettuare una copia delle configurazione degli apparati di rete. \\\\
\`{E} completamente open source, protetto da licenza GNU v3.0, ed è scritto in linguaggio PHP. Per eseguire automaticamente i task fa uso del demone Unix crontab, quindi non risulta utilizzabile su sistemi Windows, ma unicamente Linux e BSD. \\
Le connessioni verso i dispositivi non sono effettuate tramite lo standard Simple Network Management Protocol, o SNMP, ma Telnet e SSH. Questo gli permette di supportare, con la giusta configurazione, molti dispositivi, però complica notevolmente la fase di setup e può generare problemi in alcuni brand.\\\\
Questa tecnologia è stata implementata durante lo svolgimento dell'attività di stage.
\begin{comment}
\begin{figure}[H]
	\centering
	\includegraphics[width=0.34\linewidth]{"images/logo/rConfig_logo"}
	\caption{Logo di RConfig}
	\label{fig:Logo di RConfig}
\end{figure}
\end{comment}

\newpage
\section{Tecnologie per la sicurezza}
\subsection{802.1X}
Il 802.1X è uno standard IEEE per il controllo di accesso alla rete port-based, parte della famiglia di protocolli IEEE 802.1. Fornisce un meccanismo di autenticazione mediante una combinazione username/password oppure un certificato digitale. \\\\
I dispositivi utilizzati in questo protocollo sono tre: il dispositivo di un utente, che richiede l'accesso alla rete, l'Autenticator, che comunica con il dispositivo, e un Auntentication Server, che riceve la richiesta di accesso del dispositivo dall'Autenticator e decreta se approvarla o meno. \\\\
\begin{figure}[H]
	\centering
	\includegraphics[width=1\linewidth]{"images/Schema_tecnologie"}
	\caption{Schema delle iterazioni tra i dispositivi secondo 802.1X}
	\label{fig:Schema delle iterazioni tra i dispositivi secondo 802.1X}
\end{figure}

La sua caratteristica principale è l'isolamento del dispositivo dalla rete durante la fase di autenticazione, impedendogli di interagire direttamente con qualsiasi altro dispositivo che non sia l'Autenticator, il quale controllerà la validità dei messaggi ricevuti secondo il protocollo utilizzato per poi contattare il Autentication Server. \\
Questa protezione va ad ampliare la \gloss{sicurezza fisica} delle risorse connesse alla rete, pur essendo di per sè una \gloss{sicurezza logica}, impedendone il raggiungimento ai dispositivi non autorizzati. \\\\
Funziona sia su connessioni wired che wireless, anche se viene raramente supportato ed utilizzato nelle connessioni cablate in quanto, spesso, una protezione fisica perimetrale viene considerata sufficiente. \\\\
Il 802.1X determina unicamente la procedura di autenticazione che deve essere svolta, mentre la definizione degli standard dei messaggi e del trasporto viene definita da altri standard, ad esempio EAP per il contenuto dei messaggi ed EAPOL e RADIUS per il loro trasporto.

\begin{figure}[H]
	\centering
	\includegraphics[width=0.78\linewidth]{"images/8021x-Authentication-message-flow"}
	\caption{Procedura di autenticazione 802.1X}
	\label{fig:Procedura di autenticazione 802.1X}
\end{figure}


\subsection{Extensible Authentication Protocol}
Il Extensible Authentication Protocol, usualmente chiamato semplicemente EAP, è un framework di autenticazione che definisce la struttura dei messaggi. \\
Lo standard include svariate tipologie di messaggio, in modo tale da poter fornire molteplici metodologie di login, come ad esempio combinazioni username e password, utilizzo di certificati o mediante un segreto condiviso. \\\\
Non essendo un protocollo di rete ha la necessità di essere incapsulato in un messaggio per poter essere inviato, ad esempio all'interno del protocollo \gloss{RADIUS} oppure nel protocollo di rete dedicato \gloss{EAPoL}.

\subsection{Windows Network Policy Server}
Network Policy Server, spesso abbreviato in NPS, in italiano "Server dei criteri di rete", è una tecnologia per il controllo dell'accesso alla rete. \\
Decreta chi può accedere alla rete ponendo delle restrizioni ed utilizzare il protocollo \gloss{RADIUS} per comunicare. \\\\
Un server che lo implementa può decretare indipendentemente l'accesso o meno dei vari dispositivi, costituendo quindi un RADIUS server, oppure può inoltrare la richiesta ad un altro server NPS, quindi coprendo il ruolo di RADIUS proxy.

\subsection{Active Directory}
Active Directory è un insieme di servizi di rete adottati dai sistemi operativi Microsoft. \\
Sono gestiti da un \gloss{Domain Controller} e definisce le modalità di accesso alle risorse da parte degli utenti. \\
Le risorse possono essere account utente, account relativi a un computer, cartelle condivise, stampanti e servizi di varia natura. I privilegi di accesso alle varie risorse sono determinati mediante dei criteri di gruppo.

\subsection{CAPsMAN}
CAPsMAN, per esteso Controlled Access Point system MANager, traducibile in "sistema di gestione di access point controllati", è una funzionalità per la gestione degli accessi fornita da MikroTik assieme ai suoi access point. \\
La sua funzionalità principale è la propagazione di una configurazione comune per l'accesso Wireless a tutti gli apparati presenti nella rete. \\\\
Consente l'identificazione dei client che si connettono mediante un server RADIUS e permette anche di reindirizzare dinamicamente gli utenti su VLAN diverse. \\\\
Risulta molto utile in caso di estensione della rete, in quanto riduce notevolmente la configurazione iniziale di ogni access point, diminuendo a sua volta il rischio di commettere errori di configurazione.


\begin{comment}
\subsection{QRadar}
Il software QRadar, prodotto da IBM, consente la rilevazione di anomalie all'interno di una rete, sfruttando l'intelligenza artificiale per la rimozione dei falsi positivi e l'individuazione di comportamenti inaspettati.\\
Permette la raccolta, l'aggregazione e l'analisi di file di log da migliaia di dispositivi, in modo da non lasciare scoperto nessun possibile accesso non autorizzato. \\
Questo software risulta essenziale per coloro che si occupano di quest'ambito di sicurezza, in quanto la lettura e l'incrocio dei dati ottenuti dai log di tutti i dispositivi risulta esosa, se non impossibile, per un singolo team di persone.
\begin{figure}[H]
\centering
\includegraphics[width=0.5\linewidth]{"images/logo/QRadar_logo"}
\caption{Logo di QRadar}
\label{fig:Logo di QRadar}
\end{figure}

\subsection{BigFix}
A differenza di tutti gli altri software sopra citati, BigFix non lavora sui dispositivi di rete ma sui dispositivi degli utenti, gestendo il loro stato e soprattutto gli aggiornamenti del sistema e dei software in essi presenti. \\
Il programma quindi rende facile e veloce l'applicazione di patch di sicurezza e il controllo dei permessi degli utenti, in modo tale che eventuali vettori malevoli non possano sfruttare exploit conosciuti presenti nei software utilizzati. \\
Un altro utilizzo è la configurazione automatica di nuovi dispositivi inseriti nella rete, permettendo una installazione automatizzata di nuovi computer, evitando quindi errori di distrazione nella loro configurazione, i quali potrebbero generare problemi di sicurezza.
\begin{figure}[H]
\centering
\includegraphics[width=0.5\linewidth]{"images/logo/BigFix_logo"}
\caption{Logo di BigFix}
\label{fig:Logo di BigFix}
\end{figure}
\end{comment}

\newpage
\section{Strumenti di lavoro}

\subsection{Microsoft Visio}
Microsoft Visio è uno strumento dedito alla creazione di grafici ed organigrammi. \\
Nell'abito dello stage si è rivelato utile per la creazione della mappa della rete, in quanto facilmente utilizzabile e gestibile anche da persone senza una elevata conoscenza dell'informatica o di linguaggi dediti allo scopo.
\begin{figure}[H]
	\centering
	\includegraphics[width=0.73\linewidth]{"images/Visio_esempio"}
	\caption{Interfaccia di Visio}
	\label{fig:Interfaccia di Visio}
\end{figure}

\subsection{OpenOffice Calc}
OpenOffice Calc è un software per la gestione di fogli elettronici, che consente anche di aprire i formati gestiti da LibreOffice Calc e Microsoft Excel. \\
\'E stato impiegato per la realizzazione della lista dei dispositivi, in quanto consente una consultazione veloce e l'esportazione in svariati formati.
\begin{figure}[H]
	\centering
	\includegraphics[width=0.77\linewidth]{"images/screen_calc"}
	\caption{Interfaccia di Calc}
	\label{fig:Interfaccia di Calc}
\end{figure}

\subsection{GreaseMonkey}
GreaseMonkey è un plugin disponibile per il browser Mozilla Firefox che consente l'esecuzione di script in linguaggio JavaScript. \\
Permette anche lo store di variabili, il caricamento di librerie esterne e l'esecuzione con permessi privilegiati. \\
\`{E} stato utilizzato durante le attività di stage per automatizzare operazioni ripetitive, consentendo di avere più tempo libero da dedicare ad altre attività.
\begin{figure}[H]
	\centering
	\includegraphics[width=0.14\linewidth]{"images/logo/GreaseMonkey_logo"}
	\caption{Logo di GreaseMonkey}
	\label{fig:Logo di GreaseMonkey}
\end{figure}

\subsection{Putty}
Putty è un client SSH, Telnet e seriale combinato con un emulatore di terminale per consentire una iterazione con dispositivi remoti. \\
\`{E} stato ampliamente utilizzato per le connessioni SSH e seriale per la configurazione e il controllo dei dispositivi di rete.
\begin{figure}[H]
	\centering
	\includegraphics[width=0.62\linewidth]{"images/putty"}
	\caption{Schermata iniziale di Putty}
	\label{fig:Schermata iniziale di Putty}
\end{figure}

\subsection{Microsoft Telnet}
Il client Telnet fornito assieme al sistema operativo Microsoft Windows permette di collegarsi a dispositivi remoti tramite protocollo Telnet mediante un qualsiasi terminale. \\
\`{E} stato preferito a Putty per il fatto che risulta più veloce ed immediato da utilizzare, in quanto accessibile direttamente dalla console del sistema.

\subsection{MySQL Command-Line Tool}
La shell MySQL è in interfacciamento a linea testuale al database MySQL, che permette una iterazione con la struttura ed i dati in esso contenuto. \\
Nell'ambito del progetto è stato impiegato principalmente in concomitanza con rConfig, in quanto l'incompletezza del programma ha richiesto svariate volte una modifica manuale alla base di dati dalla quale attingeva le informazioni.

\subsection{Notepad++}
Notepad++ è un editor di file testuali disponibile per sistemi operativi Microsoft Windows.\\
La sua utilità nell'ambito del progetto è stata l'esecuzione di espressioni regolari per permettere una rapida conversione del formato dei dati, consentendo di convertire file CSV in matrici JavaScript.\\
\`{E} stato favorito rispetto ad altri editor in quanto possedevo già dimestichezza con esso.
\begin{figure}[H]
	\centering
	\includegraphics[width=0.7\linewidth]{"images/notepad++"}
	\caption{Interfaccia di Notepad++}
	\label{fig:Interfaccia di Notepad++}
\end{figure}

\subsection{Vim}
Vim è un editor di file testuali disponibili su molteplici piattaforme, di storica rilevanza nel mondo Unix. \\
Essendo spesso preinstallato, disponibile per un numero elevato di sistemi ed eseguibile da terminale si è rivelato comodo per la modifica dei file di configurazione sulle macchine virtuali, grazie anche alle sue modalità di iterazione con il testo.
\begin{figure}[H]
	\centering
	\includegraphics[width=0.75\linewidth]{"images/vim"}
	\caption{Interfaccia di Vim}
	\label{fig:Interfaccia di Vim}
\end{figure}


\subsection{FileZilla}
FileZilla Client è un software che permette il trasferimento di file in Rete attraverso il protocollo FTP. \\
Si è rilevato utile per la gestione dei file sulle macchine virtuali e sui dispositivi MikroTik, permettendo un facile trasferimento delle configurazioni e delle pagine web che si sono utilizzate durante tutto lo stage.
\begin{figure}[H]
	\centering
	\includegraphics[width=0.22\linewidth]{"images/logo/Filezilla_logo"}
	\caption{Logo di FileZilla}
	\label{fig:Logo di FileZilla}
\end{figure}


\end{document}
