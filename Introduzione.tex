\documentclass[Tesi.tex]{subfiles}

\begin{document}
\setcounter{chapter}{0}
\chapter{Contesto aziendale}

Nata nel 1987, Wintech Comumnication Factory SPA è ad oggi uno dei pochi \gloss{System Integrator} capace di vantare una lunga tradizione e un patrimonio di conoscenze nei diversi ambiti di competenza del settori ICT. \\\\
WinTech Spa svolge la propria attività in qualità di System Integrator che, grazie alla propria esperienza, competenza e creatività, trasforma le tecnologie IT di mercato in soluzioni informatiche innovative, efficienti e dal facile utilizzo.\\
\begin{figure}[H]
	\centering
	\includegraphics[width=0.9\linewidth]{"images/LogoWintech"}
	\caption{Logo di Wintech}
	\label{fig:Logo di Wintech}
\end{figure}

La società conta su una struttura di circa 80 risorse che svolgono la propria attività nelle Sedi di Padova, Milano, Bassano del Grappa e Pordenone; una precisa strategia di valorizzazione di partnership nazionali ed internazionali, le consente di superare i confini delle proprie dimensioni fruendo di collaborazioni di valore riconosciuto.\\

\newpage
\section{Dominio applicativo}
Wintech opera in un dominio molto vasto, che spazia dal Cloud alla \gloss{Digital Transformation}, per tale motivo in questa sezione verrà analizzato unicamente il dominio relativo allo stage conseguito, cioè quello del monitoraggio e della sicurezza fisica delle reti. \\

\subsection{Sviluppo della Cybersecurity}
L'ambito della Cybersecurity è sicuramente uno degli ambiti più dinamici all'interno del mondo informatico, con tipologie di attacchi che variano continuamente nel tempo. Questo può essere riscontrato analizzando i vettori di attacco, chiaramente descritti anche annualmente all'interno del rapporto Clusit. \\\\
Oramai è inutile chiedersi se si verrà attaccati, ma ci si deve chiedere solamente quando succederà e se si è pronti a difendersi ed a riparare ai possibili danni. 
Praticamente qualsiasi ente è un potenziale bersaglio ed, a volte, espone delle vulnerabilità delle quali non si preoccupa, in quanto suppone erroneamente di non essere appetibile per un eventuale attaccante. \\
A sostegno di quanto appena affermato viene utilizzato, nel mondo anglosassone, il seguente motto:
\begin{center} \textsl{Is not a matter of "if", but a matter of "when".} \end{center}
La quale può essere tradotta in italiano nel seguente modo:
\begin{center} \textsl{Non è una questione di "se", ma di "quando".} \end{center}
In conclusione bisogna sfatare l'immaginario collettivo in cui siano solo le grandi società americane, grandi brand, ad essere attaccate per attivismo. Le motivazioni sono notevolmente mutate e hanno come target qualsiasi azienda, anche le realtà della piccole e medie imprese che costituiscono il tessuto delle aziende italiane sono pesantemente bersagliate.

\subsection{La proposta di Wintech}
La proposta per consentire di mantenere un elevato livello di sicurezza all'interno delle proprie reti proposto da Wintech si compone di svariate tecnologie, che vanno ad integrarsi per permettere una protezione completa su tutti i possibili fronti di attacco. \\\\
Le tecnologie illustrate in questo documento sono una parte di quelle utilizzate all'interno dell'azienda, mirate al monitoraggio della rete, al controllo della configurazione degli apparati ed al port-based Network Access Control.

\end{document}
