\documentclass[Tesi.tex]{subfiles}

\begin{document}
\chapter{Valutazione retrospettiva}

\section{Tempo impiegato}
\begin{longtable}{|p{6cm}|p{3.5cm}|p{3.5cm}|}
	\hline
	{\bf Descrizione dell'attività} & {\bf Durata prevista} & {\bf Durata effettiva} \\
	\hline
	Analisi & 40h & \\
	\hline
	Progettazione & 80h & \\
	\hline
	Implementazione & 80h & \\
	\hline
	Test in produzione & 40h & \\
	\hline
	Tuning & 80h & \\
	\hline
	
\end{longtable}

\section{Risultati ottenuti}

\section{Vincoli del progetto}
\subsection{Vincoli tecnologici}
\subsubsection{Statistiche VLAN}
Una limitazione tecnologica attualmente presente in Observium è la sua incapacità di raggruppare le porte secondo la VLAN untagged che possiede. \\
Questo fatto, in concomitanza agli switch utilizzati che non permettevano la raccolta di statistiche dalle interfacce virtuali, comprendenti le VLAN, ha impedito di ottenere statistiche sull'utilizzo della rete suddivise per rete virtuale. \\
La limitazione non si sarebbe presentata con l'adozione di switch Cisco, in quanto essi raccolgono informazioni suddividendoli anche per VLAN, che possono poi essere raggruppati con facilità su Observium. \\
L'informazione che si sarebbe dovuta utilizzare per il ragruppamento delle porte fisiche è già presente all'interno del database dell'applicazione, ed è già possibile utilizzarla per il filtraggio degli avvisi, ma non è ancora presente il filtro di raggruppamento.

\subsection{Vincoli metodologici e di lavoro}

\subsection{Vincoli temporali}



\end{document}
