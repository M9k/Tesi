\documentclass[Realizzazione.tex]{subfiles}

\begin{document}
\section{Test in produzione}
\begin{itemize}
	\item \textbf{Periodo previsto}: dal 09/06/2018 al 13/07/2018;
	\item \textbf{Numero di ore previste}: 80h;
	% TODO completare ore
	\item \textbf{Periodo effettivo}: ;
	\item \textbf{Numero di ore effettive}: .
\end{itemize}
%TODO
\subsection{Monitoraggio eventuali anomalie, censirle, trubleshooting, idenfiticare la soluzione, trovare un workaround, implementare e testare la soluzione}
%TODO
\subsection{Aggiornare la documentazione}
%TODO
\subsection{Upgrade Observium}
Observium si è subito reso molto utile al monitoraggio della rete, questo ha portato alla decisione di acquistarne la versione professionale. \\
Le principali funzionalità offerte rispetto alla versione gratuita, definita Community, sono le seguenti:
\begin{itemize}
	\item Update e fix costanti e non a cadenza di 6 mesi;
	\item Accesso alla repository SVN;
	\item Accesso alla versione beta;
	\item Ragruppamento dei dispositivi e delle interfacce in base alle loro caratteristiche;
	\item Metriche sulla qualità del servizio;
	\item Ragruppamento delle statistiche;
	\item Indicazione della tipologia degli errori di trasmissione;
	\item Ricerca di un dispositivo tramite IP o MAC address;
	\item Supporto da parte del team di sviluppo.
\end{itemize}

\end{document}
