\documentclass[Tesi.tex]{subfiles}

\begin{document}
\chapter{Progetto di stage}
Lo scopo dell'attività di stage è l’analisi, la progettazione e l'implementazione di un sistema di monitoraggio e di nuove politiche di sicurezza relative a una LAN Campus. \\
Le conoscenze apprese durante lo svolgimento dell'attività consistono nella capacità di progettare ed implementare una soluzione per raggiungere gli obbiettivi prefissati, oltre che ad effettuare attività di tuning per perfezionarla.


\section{Pianificazione}
Le attività sono state suddivise in cinque fasi principali, le quali andranno a coprire tutta la durata del percorso formativo stabilito.

\subsection{Fase 1: Analisi}
\begin{itemize}
	\item \textbf{Periodo previsto}: dal 04/06/2018 al 08/06/2018;
	\item \textbf{Numero di ore previste}: 40h.
\end{itemize}

L'obiettivo di questa prima fase del percorso è familiarizzare con l'infrastruttura amministrata da Wintech e con i supporti hardware e software da utilizzare per la realizzazione del progetto. \\
Verranno analizzati lo schema di rete dell'infrastruttura, la lista degli apparati e delle loro caratteristiche, il sistema di sicurezza fisica e il sistema di monitoraggio presenti. 

\newpage
\subsection{Fase 2: Progettazione}
\begin{itemize}
	\item \textbf{Periodo previsto}: dal 11/06/2018 al 22/06/2018;
	\item \textbf{Numero di ore previste}: 40h.
\end{itemize}
	
Nella seconda fase si procederà alla configurazione del software di monitoraggio Observium e del software di gestione della versione rConfig. \\
Per facilitare le attività verrà definita la nomenclatura da dare ai dispositivi.\\
Inoltre si procederà alla definizione delle logiche di sicurezza che verranno implementate basate sul protocollo 802.11X. \\
Durante questa fase verrà anche prodotta la prima bozza della documentazione progettuale.
	

\subsection{Fase 3: Implementazione}
\begin{itemize}
	\item \textbf{Periodo previsto}: dal 25/06/2018 al 06/07/2018;
	\item \textbf{Numero di ore previste}: 80h.
\end{itemize}
	
In questa fase verrà implementata la nuova infrastruttura e saranno resi operativi i software precedentemente configurati. \\
Verrà creato un laboratorio con uno nuovo switch sul quale verranno testate le politiche precedentemente scelte per valutarne l'efficacia.\\\\
Una volta accertata la validità del nuovo sistema si procederà alla creazione delle configurazioni per tutti gli switch comprendenti la nuova politica di sicurezza e la loro installazione. \\
Successivamente i dispositivi saranno inseriti all'interno del DNS e verranno resi operativi i software Observium e RConfig. \\
Durante questo periodo la documentazione progettuale verrà aggiornata di conseguenza alle attività svolte.


\subsection{Fase 4: Test in produzione}
\begin{itemize}
	\item \textbf{Periodo previsto}: dal 09/06/2018 al 13/07/2018;
	\item \textbf{Numero di ore previste}: 40h.
\end{itemize}
	
La fase di test in produzione è la più importante perché verrà messo alla prova l'intero sistema con il picco dell'utenza. \\
L'attività che verrà svolta sarà il monitoraggio di eventuali anomalie, per poi censirle, cercare la fonte del problema, identificare una soluzione ed implementarla. \\
Durante questo periodo la documentazione progettuale verrà aggiornata di conseguenza alle attività svolte.

	

\subsection{Fase 5: Tuning}
\begin{itemize}
	\item \textbf{Periodo previsto}: dal 16/07/2018 al 27/07/2018;
	\item \textbf{Numero di ore previste}: 80h.
\end{itemize}
	
Nella fase finale del progetto verranno eseguiti tuning su tutta la rete, ove possibile, sia sugli apparati che nelle configurazioni dei software. \\
Questa attività sarà supportata dal software Observium, che nel frattempo avrà raccolto una mole di dati tale da permettere uno studio dei miglioramenti effettuabili.\\\\
Un'altra attività derivante dallo studio dei dati raccolti sarà la configurazione delle soglie di allarme automatici, che verranno comunicati tramite messaggio E-mail e Telegram. \\
In questa ultima fase si procederà anche a completare la documentazione. 

\newpage
\section{Obiettivi}
\subsection{Obiettivo aziendale}
L'obiettivo a fine stage è aumentare la sicurezza fisica e consentire il monitoraggio dello stato di una rete Campus Area Network dove accedono migliaia di persone. \\\\
La finalità è rilevare eventuali anomalie e colli di bottiglia presenti ed impedire l'uso illecito e non controllato dei servizi di rete, manipolazione delle informazioni e furto di dati. \\
Questo comprende l'implementazione e l'utilizzo di servizi e protocolli avanzati per ottenere e mantenere le migliori performance possibili garantendo anche una sicurezza elevata.

\subsection{Obiettivo formativo}
L'obiettivo per il tirocinante è acquisire competenze in ambito networking e security in un contesto reale quale una Campus Area Network estesa, variegata e con un numero di utenti elevato. \\
Questo al fine di permettere di mettere in pratica le conoscenze acquisite durante lo svolgimento dei corsi universitari e fornire uno stimolo ad approfondire ancora di più queste tematiche.

\newpage
\subsection{Risultati attesi}
Di seguito sono indicati i risultati attesi, che dovranno essere soddisfatti durante le attività di stage.

\subsubsection{Monitoraggio}
\label{table:Risultati attesi per l'implementazione di Observium}
\rowcolors{2}{CRighePari}{CRigheDispari}
\renewcommand*{\arraystretch}{1.2}
\begin{longtable}[H]{p{8cm}p{2cm}p{2cm}}
	\rowcolor{CHeader}
	\color{CHeaderText} \textbf{Misurazione} & \color{CHeaderText} \textbf{Valore minimo accettato} & \color{CHeaderText} \textbf{Valore massimo accettato} \\
	\endhead
	Tempo medio di composizione di una pagina
	& 0ms & 500ms \\
	Numero di dispositivi di rete monitorati in contemporanea
	& 80 & - \\
	Numero di interfacce monitorate in contemporanea
	& 2500 & - \\
	Numero di sensori e periferiche monitorati in contemporanea
	& 500 & - \\
	Numero di alert checks presenti
	& 10 & 20 \\
	Protezione delle informazioni tramite password
	& Si & - \\
	\hiderowcolors
	\caption{Risultati attesi per l'implementazione di Observium}
\end{longtable}

\subsubsection{Gestione delle configurazioni dei dispositivi di rete}
\label{table:Risultati attesi per l'implementazione di rConfig}
\rowcolors{2}{CRighePari}{CRigheDispari}
\renewcommand*{\arraystretch}{1.2}
\begin{longtable}[H]{p{8cm}p{2cm}p{2cm}}
	\rowcolor{CHeader}
	\color{CHeaderText} \textbf{Misurazione} & \color{CHeaderText} \textbf{Valore minimo accettato} & \color{CHeaderText} \textbf{Valore massimo accettato} \\
	\endhead
	Tempo di composizione di una pagina
	& 0ms & 200ms \\
	Numero di dispositivi di rete gestiti in contemporanea
	& 80 & - \\
	Protezione delle informazioni tramite password
	& Si & - \\
	\hiderowcolors
	\caption{Risultati attesi per l'implementazione di rConfig}
\end{longtable}

\newpage
\subsubsection{Politiche di sicurezza}
\label{table:Risultati attesi per l'implementazione delle politiche di sicurezza}
\rowcolors{2}{CRighePari}{CRigheDispari}
\renewcommand*{\arraystretch}{1.2}
\begin{longtable}[H]{p{9.5cm}p{3.4cm}}
	\rowcolor{CHeader}
	\color{CHeaderText} \textbf{Misurazione} & \color{CHeaderText} \textbf{Valore richiesto} \\
	\endhead
	Possibilità per gli impiegati di usufruire dei servizi locali alla rete (stampanti, fax) &
	Si \\
	Possibilità per gli impiegati di usufruire dei servizi remoti (accesso ai database, relay email) &
	Si \\
	Possibilità per gli impiegati di accedere ad internet attraverso qualsiasi protocollo &
	Si \\
	Possibilità per i dispositivi mobili di usufruire dei servizi locali alla rete (stampanti, fax) &
	No \\
	Possibilità per i dispositivi mobili di usufruire dei servizi remoti (accesso ai database, relay email) &
	No \\
	Possibilità per i dispositivi mobili di accedere ad internet attraverso qualsiasi protocollo &
	No \\
	Possibilità per i dispositivi mobili di accedere ad internet attraverso i protocolli HTTP, HTTPS, SMTP, SMTPS, POP3, POP3S, IMAP, IMAPS, IPSEC VPN &
	Si \\
	Possibilità per gli ospiti di usufruire dei servizi locali alla rete (stampanti, fax) &
	No \\
	Possibilità per gli ospiti di usufruire dei servizi remoti (accesso ai database, relay email) &
	No \\
	Possibilità per gli ospiti di accedere ad internet attraverso qualsiasi protocollo &
	No \\
	Possibilità per gli ospiti di accedere ad internet attraverso i protocolli HTTP, HTTPS, SMTP, SMTPS, POP3, POP3S, IMAP, IMAPS, IPSEC VPN &
	Si \\
	\hiderowcolors
	\caption{Risultati attesi per l'implementazione delle politiche di sicurezza}
\end{longtable}

\newpage
\subsubsection{Captive Portal}
\label{table:Risultati attesi per l'implementazione del Captive Portal}
\rowcolors{2}{CRighePari}{CRigheDispari}
\renewcommand*{\arraystretch}{1.2}
\begin{longtable}[H]{p{9.5cm}p{3.4cm}}
	\rowcolor{CHeader}
	\color{CHeaderText} \textbf{Misurazione} & \color{CHeaderText} \textbf{Valore richiesto} \\
	\endhead
	Possibilità di effettuare il login &
	Si \\
	Possibilità di registrazione tramite sponsor &
	Si \\
	Registrazione valida per un determinato periodo di tempo &
	Si \\
	Interfaccia utilizzabile da un dispositivo mobile &
	Si \\
	Registrazione tramite e-mail &
	Si \\
	Possibilità di registrazione con e-mail altrui &
	No \\
	Sezione di amministrazione facilmente accessibile &
	Si \\
	Sezione di amministrazione protetta da password &
	Si \\
	Funzionalità di gestione degli sponsor &
	Si \\
	Funzionalità di gestione degli account &
	Si \\
	Funzionalità di gestione delle durate temporali &
	Si \\
	
	\hiderowcolors
	\caption{Risultati attesi per l'implementazione del Captive Portal}
\end{longtable}

\end{document}
