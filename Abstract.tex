\documentclass[Tesi.tex]{subfiles}
\begin{document}
\clearpage\thispagestyle{empty}
	
\renewcommand{\chaptername}{}
\renewcommand{\thechapter}{}
\chapter{Sommario}
Questo documento si prefigge lo scopo di presentare il lavoro svolto durante l'attività di stage, svolta presso Wintech Communications Factory. \\
Le attività sono state intraprese nell'ambito sistemistico, andando a monitorare e a migliorare la sicurezza di una Campus Area Network di dimensioni riguardevoli. \\\\

Gli argomenti trattati riguarderanno il controllo in tempo reale dello stato della rete, individuando colli di bottiglia, problematiche di varia natura e potenziali pericoli, attui a consentire un intervento tempestivo o preventivo.\\
Ulteriormente verrà trattata anche la sicurezza fisica delle reti, impedendo un accesso non autorizzato a risorse di elevata criticità a coloro che non posseggono dei privilegi sufficientemente elevati. \\\\

L'ultima parte di questo documento conterrà una mia valutazione personale retrospettiva sul lavoro svolto, nella quale valuterò i risultati formativi ed aziendali dello stage, evidenziandone i vincoli. \\

\newpage
\chapter{Convenzioni tipografiche}
Per favorire la lettura del documento e la sua comprensione questo documento presenta un glossario. \\
Qualora un termine presente nel glossario comparisse all'interno del testo, in un contesto nel quale si suppone il lettore possa non comprenderlo, verrà evidenziato rispetto al paragrafo venendo scritto in corsivo e presentando al suo termine una "G" al pedice. \\
Nel caso si stesse visionando la versione digitale di questo documento è possibile essere reindirizzati direttamente al termine nel glossario semplicemente cliccandoci sopra. \\


\clearpage
\end{document}
