\documentclass[Tesi.tex]{subfiles}

\begin{document}
\chapter{Progetto di stage}
Lo scopo dell'attività di stage è l’analisi, la progettazione e l'implementazione di un sistema di monitoraggio della sicurezza fisica di una LAN Campus. \\
Le conoscenze apprese durante lo svolgimento dell'attività consistono nella capacità di progettare ed implementare una soluzione per raggiungere gli obbiettivi prefissati, oltre che ad effettuare attività di tuning per perfezionarla.


\section{Pianificazione}
Le attività sono state suddivise in cinque fasi principali, le quali andranno a coprire tutta la durata del percorso formativo stabilito.

\subsection{Fase 1: Analisi}
\begin{enumerate}
	\item \textbf{Periodo previsto}: dal 04/06/2018 al 08/06/2018;
	\item \textbf{Numero di ore previste}: 40h;
	% TODO completare ore
	\item \textbf{Periodo effettivo}: ;
	\item \textbf{Numero di ore effettive}: .
\end{enumerate}

L'obiettivo di questa prima fase del percorso è familiarizzare con l'infrastruttura amministrata da Wintech e con i supporti hardware e software da utilizzare per la realizzazione del progetto. \\
Nello specifico è stato analizzato lo schema di rete dell'infrastruttura, la lista degli apparati e delle loro caratteristiche, il sistema di sicurezza fisica attivi e il sistema di monitoraggio presenti. 

\subsection{Fase 2: Progettazione}
\begin{enumerate}
	\item \textbf{Periodo previsto}: dal 11/06/2018 al 22/06/2018;
	\item \textbf{Numero di ore previste}: 40h;
	% TODO completare ore
	\item \textbf{Periodo effettivo}: ;
	\item \textbf{Numero di ore effettive}: .
	
	%TODO descrizione
	
\end{enumerate}

\subsection{Fase 3: Implementazione}
\begin{enumerate}
	\item \textbf{Periodo previsto}: dal 25/06/2018 al 06/07/2018;
	\item \textbf{Numero di ore previste}: 80h;
	% TODO completare ore
	\item \textbf{Periodo effettivo}: ;
	\item \textbf{Numero di ore effettive}: .
	
	%TODO descrizione
	
\end{enumerate}

\subsection{Fase 4: Test in produzione}
\begin{enumerate}
	\item \textbf{Periodo previsto}: dal 09/06/2018 al 13/07/2018;
	\item \textbf{Numero di ore previste}: 80h;
	% TODO completare ore
	\item \textbf{Periodo effettivo}: ;
	\item \textbf{Numero di ore effettive}: .
	
	%TODO descrizione
	
\end{enumerate}

\subsection{Fase 5: Tuning}
\begin{enumerate}
	\item \textbf{Periodo previsto}: dal 16/07/2018 al 27/07/2018;
	\item \textbf{Numero di ore previste}: 80h;
	% TODO completare ore
	\item \textbf{Periodo effettivo}: ;
	\item \textbf{Numero di ore effettive}: .
	
	%TODO descrizione
	
\end{enumerate}

\section{Riepilogo ore}
    \begin{longtable}{|p{3cm}|p{10cm}|}
	\hline
	{\bf Durata in ore} & {\bf Descrizione dell'attività} \\
	\hline
	40 & Analisi \\
	\hline
	80 & Progettazione \\
	\hline
	80 & Implementazione \\
	\hline
	40 & Test in produzione \\
	\hline
	80 & Tuning \\
	\hline
	
\end{longtable}

\textbf{Numero di ore previste}: 320h

\section{Obiettivi} % separare quelli formativi da quelli per l'azienda?

\end{document}
