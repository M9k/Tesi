\documentclass[Tesi.tex]{subfiles}

\begin{document}
\chapter{Tecnologie}

Questa sezione illustra le principali tecnologie utilizzate da Wintech S.P.A per il controllo ed il monitoraggio delle reti di loro competenza. \\
Essendo la sicurezza informatica un settore nel quale gli standard, le leggi e le tipologie di attacco evolvono molto in fretta, le tecnologie utilizzate devono adeguarsi di conseguenza. Questo porta i tool ed i programmi utilizzati a diventare obsoleti oppure variare con il passare del tempo.
%TODO riferimenti?
%TODO anche QRadar e BigFix?
\subsection{802.1X}
Il 802.1X è uno standard IEEE per il controllo di accesso alla rete port-based, parte della famiglia di protocolli IEEE 802.1. Fornisce un meccanismo di autenticazione mediante una combinazione username/password oppure un certificato digitale.
Questa protezione va ad ampliare la \gloss{sicurezza fisica} delle risorse connesse alla rete, pur essendo di per sè una \gloss{sicurezza logica}, impedendone il raggiungimento ai dispositivi non autorizzati. \\
Lo standard viene implementato all'interno del firmware o del sistema operativo degli apparati di rete e dei dispositivi degli utenti.\\
Funziona sia su connessioni wired che wireless, anche se viene raramente utilizzato nelle connessioni cablate in quanto, ove possibile, viene favorita una protezione fisica perimetrale, apparentemente migliore.

\subsection{PRTG Network Monitor}
PRTG è un sistema di monitoraggio della rete, era utilizzato nella rete analizzata durante lo stage prima di Observium per rilevare eventuali anomalie, ma a causa delle limitazioni della versione gratuita e del suo elevato costo si è scelto di sostituirlo.\\
Il software consiste in un servizio al quale ci si collega mediante il client fornito oppure attraverso una interfaccia web e permette di tenere sotto controllo anche siti web e servizi di varia natura. \\
Alla fine dell'attività di stage il software non è stato disabilitato, ma gran parte dei suoi compiti sono ora assolti da Observium, in quanto possiede alcune funzionalità che il suo concorrente non offre. \\
\begin{figure}[H]
	\centering
	\includegraphics[width=0.5\linewidth]{"images/logo/PRTG_logo"}
	\caption{Logo di PRTG}
	\label{fig:Logo di PRTG}
\end{figure}

\subsection{Observium}
Observium è un sistema di monitoraggio della rete compatibile con i dispositivi delle principali aziende produttrici di apparati di rete. \\
Tra i dispositivi supportati si possono citare Cisco, Dell, HP, Huawei, Lenovo, MikroTik, Netgear e ZTE. \\
Il software dispone anche di una ricerca automatica dei dispositivi presente all'interno della rete, utile per reti di piccola-media dimensione. \\
Le funzionalità che fornisce sono il monitoraggio del \gloss{QOS}, il raggruppamento dei dispositivi mediante regole definite dall'utente e l'alert automatico nel caso vengano superate determinate soglie.
\begin{figure}[H]
	\centering
	\includegraphics[width=0.5\linewidth]{"images/logo/Observium_logo"}
	\caption{Logo di Observium}
	\label{fig:Logo di Observium}
\end{figure}

\subsection{RConfig}
Il tool RConfig è un configuration management mirato ai dispositivi di rete che permette in modo veloce ed automatizzato di effettuare una copia delle configurazione degli apparati di rete. \\
\`{E} completamente open source, protetto da licenza GNU v3.0, ed è scritto in linguaggio PHP, questo gli consente di essere installato facilmente su molti sistemi.
\begin{figure}[H]
	\centering
	\includegraphics[width=0.5\linewidth]{"images/logo/rConfig_logo"}
	\caption{Logo di RConfig}
	\label{fig:Logo di RConfig}
\end{figure}

\subsection{GreaseMonkey}
GreaseMonkey è un plugin disponibile per il browser Mozilla Firefox che consente l'esecuzione di script in linguaggio JavaScript. \\
Permette anche lo store di variabili, il caricamento di librerie esterne e l'esecuzione con permessi privilegiati. \\
\`{E} stato utilizzato durante le attività di stage per automatizzare operazioni ripetitive, consentendo di impiegarci meno tempo, da dedicare ad altre attività.
\begin{figure}[H]
	\centering
	%TODO: logo greasemonkey
	\includegraphics[width=0.5\linewidth]{"images/logo/rConfig_logo"}
	\caption{Logo di GreaseMonkey}
	\label{fig:Logo di GreaseMonkey}
\end{figure}

\begin{comment}
\subsection{QRadar}
Il software QRadar, prodotto da IBM, consente la rilevazione di anomalie all'interno di una rete, sfruttando l'intelligenza artificiale per la rimozione dei falsi positivi e l'individuazione di comportamenti inaspettati.\\
Permette la raccolta, l'aggregazione e l'analisi di file di log da migliaia di dispositivi, in modo da non lasciare scoperto nessun possibile accesso non autorizzato. \\
Questo software risulta essenziale per coloro che si occupano di quest'ambito di sicurezza, in quanto la lettura e l'incrocio dei dati ottenuti dai log di tutti i dispositivi risulta esosa, se non impossibile, per un singolo team di persone.
\begin{figure}[H]
\centering
\includegraphics[width=0.5\linewidth]{"images/logo/QRadar_logo"}
\caption{Logo di QRadar}
\label{fig:Logo di QRadar}
\end{figure}

\subsection{BigFix}
A differenza di tutti gli altri software sopra citati, BigFix non lavora sui dispositivi di rete ma sui dispositivi degli utenti, gestendo il loro stato e soprattutto gli aggiornamenti del sistema e dei software in essi presenti. \\
Il programma quindi rende facile e veloce l'applicazione di patch di sicurezza e il controllo dei permessi degli utenti, in modo tale che eventuali vettori malevoli non possano sfruttare exploit conosciuti presenti nei software utilizzati. \\
Un altro utilizzo è la configurazione automatica di nuovi dispositivi inseriti nella rete, permettendo una installazione automatizzata di nuovi computer, evitando quindi errori di distrazione nella loro configurazione, i quali potrebbero generare problemi di sicurezza.
\begin{figure}[H]
\centering
\includegraphics[width=0.5\linewidth]{"images/logo/BigFix_logo"}
\caption{Logo di BigFix}
\label{fig:Logo di BigFix}
\end{figure}
\end{comment}




\end{document}
