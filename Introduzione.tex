\documentclass[Tesi.tex]{subfiles}

\begin{document}
\chapter{Contesto aziendale}

Nata nel 1987, Wintech Comumnication Factory SPA è ad oggi uno dei pochi \gloss{System Integrator} capace di vantare una lunga tradizione e un patrimonio di conoscenze nei diversi ambiti di competenza del settori ICT. \\
Tra le partnership si possono citare IBM, Symantec, Hewlett Packard, Microsoft, Oracle, VMWare e Sophos.


\section{Dominio applicativo}
Wintech opera in un dominio molto vasto, che spazia dal Cloud alla \gloss{Digital Transformation}, per tale motivo in questa sezione verrà analizzato unicamente il dominio relativo allo stage conseguito, cioè quello del monitoraggio e della sicurezza fisica delle reti. \\

\subsection{Com'è cambiata la Cybersecurity negli ultimi anni?}
Il panorama attuale per tipologia di attacchi e motivazioni, è profondamente mutato rispetto anche solo a qualche anno fa. Tutti gli osservatori sulla tematica Cybersecurity lo confermano con i dati. Non è fare terrorismo psicologico, ma essere realisti, quando si afferma che tutti si è potenzialmente vulnerabili e bersagli del cybercrime e quindi "is not a matter of if, but a matter of when". Bisogna sfatare l’immaginario collettivo in cui siano solo le grandi società americane, grandi brand, ad essere attaccate per attivismo. Le motivazioni sono notevolmente mutate e hanno come target qualsiasi azienda, anche le realtà della piccole e medie imprese che costituiscono il tessuto delle aziende italiane sono pesantemente bersagliate.

\subsection{La proposta di Wintech}
La proposta per consentire di mantenere un elevato livello di sicurezza all'interno delle proprie reti proposto da Wintech si compone di svariate tecnologie, che vanno ad 
%TODO incastrarsi suona male
incastrarsi
per permettere una protezione completa su tutti i possibili fronti di attacco. \\
%TODO controllare con il senno di poi
Le tecnologie proposte saranno riguardanti il monitoraggio della rete, il controllo della configurazione degli apparati ed il port-based Network Access Control.


\section{Tecnologie e standard utilizzati}
Questa sezione illustra le principali tecnologie utilizzate da Wintech S.P.A per il controllo ed il monitoraggio delle reti di loro competenza. \\
Essendo la sicurezza informatica un settore nel quale gli standard, le leggi e le tipologie di attacco evolvono molto in fretta, le tecnologie utilizzate devono adeguarsi di conseguenza. Questo porta i tool ed i programmi utilizzati a diventare obsoleti oppure variare con il passare del tempo.
%TODO riferimenti?
%TODO anche QRadar e BigFix?
\subsection{802.1X}
Il 802.1X è uno standard IEEE per il controllo di accesso alla rete port-based, parte della famiglia di protocolli IEEE 802.1.\\
Fornisce un meccanismo di autenticazione mediante una combinazione username/password oppure un certificato digitale.\\
Questo standard viene implementato all'interno del firmware o del sistema operativo degli apparati di rete e dei dispositivi degli utenti.\\
Funziona sia su connessioni wired che wireless, anche se viene raramente utilizzato nelle connessioni cablate in quanto, ove possibile, viene favorita una protezione fisica perimetrale, apparentemente migliore.

\subsection{Observium}
Observium è un sistema di monitoraggio della rete compatibile con i dispositivi delle principali aziende produttrici di apparati di rete. \\
Tra i dispositivi supportati si possono citare Cisco, Dell, HP, Huawei, Lenovo, MikroTik, Netgear e ZTE. \\
Il software dispone anche di una ricerca automatica dei dispositivi presente all'interno della rete, utile per reti di piccola-media dimensione. \\
Le funzionalità che fornisce sono il monitoraggio del \gloss{QOS}, il raggruppamento dei dispositivi mediante regole definite dall'utente e l'alert automatico nel caso vengano superate determinate soglie.

\subsection{RConfig}
Il tool RConfig è un configuration management mirato ai dispositivi di rete che permette in modo veloce ed automatizzato di effettuare una copia delle configurazione degli apparati di rete. \\
\`{E} completamente open source, protetto da licenza GNU v3.0, ed è scritto in linguaggio PHP, questo gli consente di essere installato facilmente su molti sistemi.

\subsection{QRadar}
Il software QRadar, prodotto da IBM, consente la rilevazione di anomalie all'interno di una rete, sfruttando l'intelligenza artificiale per la rimozione dei falsi positivi e l'individuazione di comportamenti inaspettati.\\
Permette la raccolta, l'aggregazione e l'analisi di file di log da migliaia di dispositivi, in modo da non lasciare scoperto nessun possibile accesso non autorizzato. \\
Questo software risulta essenziale per coloro che si occupano di quest'ambito di sicurezza, in quanto la lettura e l'incrocio dei dati ottenuti dai log di tutti i dispositivi risulta esosa, se non impossibile, per un singolo team di persone.

\subsection{BigFix}
A differenza di tutti gli altri software sopra citati, BigFix non lavora sui dispositivi di rete ma sui dispositivi degli utenti, gestendo il loro stato e soprattutto gli aggiornamenti del sistema e dei software in essi presenti. \\
Il programma quindi rende facile e veloce l'applicazione di patch di sicurezza e il controllo dei permessi degli utenti, in modo tale che eventuali vettori malevoli non possano sfruttare exploit conosciuti presenti nei software utilizzati. \\
Un altro utilizzo è la configurazione automatica di nuovi dispositivi inseriti nella rete, permettendo una installazione automatizzata di nuovi computer, evitando quindi errori di distrazione nella loro configurazione, i quali potrebbero generare problemi di sicurezza.

\section{Tipologia di clientela aziendale}



\section{Processi aziendali}


\end{document}
