\documentclass[Tesi.tex]{subfiles}

\begin{document}
\chapter{Progetto di stage}
Lo scopo dell'attività di stage è l’analisi, la progettazione e l'implementazione di un sistema di monitoraggio e di nuove politiche di sicurezza relative a una Campus Area Network, impropriamente detta anche LAN Campus. \\
Le conoscenze apprese durante lo svolgimento dell'attività consistono nella capacità di progettare e implementare una soluzione per raggiungere gli obiettivi prefissati, oltre che ad effettuare attività di tuning per perfezionarla.


\section{Panoramica del progetto}

Il progetto di stage è costituito principalmente da tre parti:
\begin{itemize}
	\item Sistema di monitoraggio;
	\item Sistema di versionamento delle configurazioni;
	\item Nuove politiche di sicurezza fisica.
\end{itemize}

Il sistema di monitoraggio avrà il compito di mantenere costantemente la rete sotto osservazione, notificando quando si presentano problemi. \\
Il sistema di versionamento delle configurazioni degli apparati di rete deve permettere una sostituzione fulminea in caso di guasti dell'infrastruttura, consentendo di ripristinare tutte le impostazioni presenti senza doverle ridefinire nuovamente. \\
Per concludere verrà sviluppato un sistema di autenticazione ad una rete WiFi che impedisca un accesso fisico alle risorse collegate per coloro che non ne possiedono i relativi privilegi.\\

\subsection{Obiettivi dello stage}
Il progetto di stage ha come obiettivo quello di risolvere alcune problematiche riscontrate dal cliente all'interno della propria rete. \\\\
Il primo problema è la presenza di un controllo non soddisfacente della rete, che non consente di individuare svariate problematiche finché esse non degenerano in un disservizio. \\\\
Anche le procedure di disaster recovery in caso di guasti ai dispositivi non erano adeguate, in quanto non esisteva nessun backup o indicazione della configurazione utilizzata, quindi in caso di guasto si doveva configurare da zero il nuovo apparato, identificando le periferiche connesse seguendo il cablaggio.\\\\
L'ultima problematica individuata era la rete Wireless presente negli uffici, protetta unicamente da una unica password  che veniva a volte fornita agli ospiti e, senza autorizzazione, utilizzata anche per i dispositivi personali dei dipendenti. La situazione veniva aggravata dal fatto che la rete in esame fornisce l'accesso a risorse di elevata importanza, che non possono essere isolate in quanto utilizzate per lo svolgimento dell'attività lavorativa.

\subsection{Soluzioni e strumenti utilizzati}
Per implementare il monitoraggio si è scelto un software denominato Observium, che è andato ad affiancare ed in larga parte a sostituire un servizio precedentemente utilizzato denominato PRTG Monitor. \\

Il software scelto per la raccolta automatica delle configurazioni è stato RConfig, che però ha presentato alcune problematiche nella sua implementazione con i dispositivi MikroTik. Questo ha richiesto lo sviluppo di uno script aggiuntivo da caricare sui dispositivi stessi per poter raggiungere l'obiettivo prefissato. \\

Per garantire una sicurezza fisica si è scelto di usufruire del protocollo di autenticazione 802.1X, sviluppato appositamente per questa finalità.\\
La rete WiFi è stata implementata utilizzando il protocollo CAPsMAN in modo tale da favorirne l'estensione e la gestione, collegata a dei servizi NPS e FreeRADIUS per verificare l'autenticazione degli utenti. \`{E} stato necessario anche sviluppare un piccolo Captive Portal al fine di consentire la registrazione ad eventuali ospiti.

\subsection{Problemi riscontrati}
Durante l'attività di stage sono state incontrate alcune difficoltà e limitazioni tecnologiche, che hanno protratto il tempo impiegato per alcune attività. La più critica è stata l'impossibilità per RConfig di leggere autonomamente la configurazione dei dispositivi prodotti da MikroTik, che ha richiesto uno sviluppo di uno script dedicato non previsto inizialmente.

\subsection{Risultati ottenuti}
Al termine dell'attività di stage tutte le problematiche inizialmente rilevate sono state risolte. \\
Il sistema di monitoraggio mediante Observium e il versionamento delle configurazioni realizzato con RConfig e script MikroTik sono stati completati con successo, permettendo quindi di prevenire eventuali problematiche e, nel caso che ciò non fosse possibile, intervenire con rapidità.\\
Le modalità di accesso alla rete WiFi degli uffici è stata completamente riprogettata, suddividendo gli utenti in tre gruppi in base al loro ruolo, i quali presentano modalità di autenticazioni diverse. Le risorse sono state rese disponibili unicamente per la categoria di persone che ne hanno necessità e sono protette dallo standard 802.1X, che ne garantisce la segregazione fino ad una corretta autenticazione dell'utente.

\section{Struttura del documento di relazione}
Questo documento è stato strutturato in più parti, che analizzano e ripercorrono tutto il progetto.\\\\
La prima parte introduce l'azienda presso la quale si è svolta l'attività di stage e il progetto di stage a grandi linee. \\\\
Successivamente viene presentata la pianificazione delle attività e gli obiettivi che si mira a raggiungere, analizzando anche le tecnologie utilizzate per raggiungerli. \\
La parte più consistente è la realizzazione, che descrive i compiti conseguiti per il soddisfacimento degli obiettivi prefissati. \\\\
La relazione si conclude con una valutazione retrospettiva dell'attività, nella quale viene analizzato lo scostamento temporale rispetto alla pianificazione, vengono presentati i risultati ottenuti e i vincoli che ho incontrato.


\chapter{Pianificazione}
Le attività sono state suddivise in cinque fasi principali, le quali hanno coperto tutta la durata del percorso formativo stabilito.

\section{Fase 1: Analisi}
\begin{itemize}
	\item \textbf{Periodo previsto}: dal 04/06/2018 al 08/06/2018;
	\item \textbf{Numero di ore previste}: 40h.
\end{itemize}

L'obiettivo di questa prima fase è stata la familiarizzazione con l'infrastruttura amministrata da Wintech e con i supporti hardware e software da utilizzare per la realizzazione del progetto. \\
Sono stati analizzati lo schema di rete dell'infrastruttura, il sistema di sicurezza fisica, il software di monitoraggio presente, la lista degli apparati e le loro caratteristiche. 

\section{Fase 2: Progettazione}
\begin{itemize}
	\item \textbf{Periodo previsto}: dal 11/06/2018 al 22/06/2018;
	\item \textbf{Numero di ore previste}: 40h.
\end{itemize}
	
Nella seconda fase si è proceduto alla configurazione del software di monitoraggio Observium e del software di gestione della versione rConfig. \\
Per facilitare le attività è stata definita la nomenclatura da dare ai dispositivi.\\
Inoltre si è proceduto alla definizione delle logiche di sicurezza basate sul protocollo 802.11X. \\
Durante questa fase è anche stata prodotta la prima bozza della documentazione progettuale.
	

\section{Fase 3: Implementazione}
\begin{itemize}
	\item \textbf{Periodo previsto}: dal 25/06/2018 al 06/07/2018;
	\item \textbf{Numero di ore previste}: 80h.
\end{itemize}
	
In questa fase è stata implementata la nuova infrastruttura e sono stati resi operativi i software individuati nella fase precedente. \\
\`{E} stato creato un laboratorio con uno nuovo switch sul quale sono state testate le politiche precedentemente scelte per valutarne l'efficacia.\\\\
Una volta accertata la validità del nuovo sistema si è proceduto alla creazione e all'installazione delle configurazioni relative alle nuove politiche di sicurezza. \\
Successivamente i dispositivi sono stati inseriti all'interno del DNS e si sono resi operativi i software Observium e RConfig. \\
Durante questo periodo la documentazione progettuale è aggiornata in conseguenza alle attività svolte.


\section{Fase 4: Test}
\begin{itemize}
	\item \textbf{Periodo previsto}: dal 09/06/2018 al 13/07/2018;
	\item \textbf{Numero di ore previste}: 40h.
\end{itemize}
	
La fase di test è la più importante perché è stato messo alla prova l'intero sistema con il picco dell'utenza. \\
L'attività svolta è stata il monitoraggio di eventuali anomalie, per poi censirle, cercare la fonte del problema, identificare una soluzione e implementarla. \\
Durante questo periodo la documentazione progettuale è aggiornata in conseguenza alle attività svolte.

	

\section{Fase 5: Tuning}
\begin{itemize}
	\item \textbf{Periodo previsto}: dal 16/07/2018 al 27/07/2018;
	\item \textbf{Numero di ore previste}: 80h.
\end{itemize}
	
Nella fase finale del progetto sono stati eseguiti tuning su tutta la rete, ove possibile, sia sugli apparati che nelle configurazioni dei software. \\
Questa attività è stata supportata dal software Observium, che nel frattempo ha raccolto una mole di dati tale da permettere uno studio dei miglioramenti effettuabili.\\\\
Un'altra attività derivante dallo studio dei dati raccolti è stata la configurazione delle soglie di allarme automatici, i quali vengono comunicati tramite messaggio E-mail e Telegram. \\
In questa ultima fase si è proceduto anche a completare la documentazione. 

\newpage
\chapter{Obiettivi}
\section{Obiettivo aziendale}
L'obiettivo a fine stage è aumentare la sicurezza fisica e consentire il monitoraggio dello stato di una rete Campus Area Network dove accedono migliaia di persone. \\\\
La finalità è rilevare eventuali anomalie e colli di bottiglia presenti ed impedire l'uso illecito e non controllato dei servizi di rete, manipolazione delle informazioni e furto di dati. \\
Questo comprende l'implementazione e l'utilizzo di servizi e protocolli avanzati per ottenere e mantenere le migliori performance possibili garantendo anche una sicurezza elevata.

\section{Obiettivo formativo}
L'obiettivo per il tirocinante è acquisire competenze in ambito networking e security in un contesto reale quale una Campus Area Network estesa, variegata e con un numero di utenti elevato. \\
Questo al fine di permettere di mettere in pratica le conoscenze acquisite durante lo svolgimento dei corsi universitari e fornire uno stimolo ad approfondire ancora di più queste tematiche.

\newpage
\section{Risultati attesi}
Di seguito sono indicati i risultati attesi, che dovranno essere soddisfatti durante le attività di stage. \\
Questi valori sono stati scelti in comune accordo tra il tutor aziendale ed il cliente, considerando la natura e l'estensione della rete.

\subsection{Monitoraggio}
\label{table:Risultati attesi per l'implementazione di Observium}
\rowcolors{2}{CRighePari}{CRigheDispari}
\renewcommand*{\arraystretch}{1.2}
\begin{longtable}[H]{p{8cm}p{2cm}p{2cm}}
	\rowcolor{CHeader}
	\color{CHeaderText} \textbf{Misurazione} & \color{CHeaderText} \textbf{Valore minimo accettato} & \color{CHeaderText} \textbf{Valore massimo accettato} \\
	\endhead
	Tempo medio di composizione di una pagina
	& 0ms & 500ms \\
	Numero di dispositivi di rete monitorati in contemporanea
	& 80 & - \\
	Numero di interfacce monitorate in contemporanea
	& 2500 & - \\
	Numero di sensori e periferiche monitorati in contemporanea
	& 500 & - \\
	Numero di alert checks presenti
	& 10 & 20 \\
	Protezione delle informazioni tramite password
	& Si & - \\
	\hiderowcolors
	\caption{Risultati attesi per l'implementazione di Observium}
\end{longtable}
Il numero di dispositivi presenti sono stati decretati in modo tale da riuscire a monitorare efficacemente la rete presente.\\
Gli alert check individuano le situazioni considerate pericolose, il loro numero è stato scelto in modo tale da permettere una individuazione precisa della problematica senza però aggiungere troppa verbosità.

\newpage
\subsection{Gestione delle configurazioni dei dispositivi di rete}
\label{table:Risultati attesi per l'implementazione di rConfig}
\rowcolors{2}{CRighePari}{CRigheDispari}
\renewcommand*{\arraystretch}{1.2}
\begin{longtable}[H]{p{8cm}p{2cm}p{2cm}}
	\rowcolor{CHeader}
	\color{CHeaderText} \textbf{Misurazione} & \color{CHeaderText} \textbf{Valore minimo accettato} & \color{CHeaderText} \textbf{Valore massimo accettato} \\
	\endhead
	Tempo di composizione di una pagina
	& 0ms & 200ms \\
	Numero di dispositivi di rete gestiti in contemporanea
	& 80 & - \\
	Protezione delle informazioni tramite password
	& Si & - \\
	\hiderowcolors
	\caption{Risultati attesi per l'implementazione di rConfig}
\end{longtable}
Il tempo di composizione di una pagina concesso è inferiore rispetto a quello del monitoraggio in quanto le informazioni presenti all'interno del sistema sono inferiori numericamente, quindi il tempo necessario per visualizzarle dovrà essere inferiore.


\subsection{Politiche di sicurezza}
\label{table:Risultati attesi per l'implementazione delle politiche di sicurezza}
\rowcolors{2}{CRighePari}{CRigheDispari}
\renewcommand*{\arraystretch}{1.2}
\begin{longtable}[H]{p{9.5cm}p{3.4cm}}
	\rowcolor{CHeader}
	\color{CHeaderText} \textbf{Misurazione} & \color{CHeaderText} \textbf{Valore richiesto} \\
	\endhead
	Possibilità per gli impiegati di usufruire dei servizi locali alla rete (stampanti, fax) &
	Si \\
	Possibilità per gli impiegati di usufruire dei servizi remoti (accesso ai database, relay email) &
	Si \\
	Possibilità per gli impiegati di accedere ad internet attraverso protocolli definiti &
	Si \\
	Possibilità per i dispositivi mobili del personale di usufruire dei servizi locali alla rete (stampanti, fax) &
	No \\
	Possibilità per i dispositivi mobili del personale di usufruire dei servizi remoti (accesso ai database, relay email) &
	No \\
	Possibilità per i dispositivi mobili del personale di accedere ad internet attraverso qualsiasi protocollo &
	No \\
	Possibilità per i dispositivi mobili del personale di accedere ad internet attraverso i protocolli HTTP, HTTPS, SMTP, SMTPS, POP3, POP3S, IMAP, IMAPS, IPSEC VPN &
	Si \\
	Necessità per i dispositivi mobili di effettuare una autenticazione tramite Captive Portal &
	No \\
	Possibilità per gli ospiti di usufruire dei servizi locali alla rete (stampanti, fax) &
	No \\
	Possibilità per gli ospiti di usufruire dei servizi remoti (accesso ai database, relay email) &
	No \\
	Possibilità per gli ospiti di accedere ad internet attraverso qualsiasi protocollo &
	No \\
	Possibilità per gli ospiti di accedere ad internet attraverso i protocolli HTTP, HTTPS, SMTP, SMTPS, POP3, POP3S, IMAP, IMAPS, IPSEC VPN &
	Si \\
	Necessità per gli ospiti di effettuare una autenticazione tramite Captive Portal &
	Si \\
	\hiderowcolors
	\caption{Risultati attesi per l'implementazione delle politiche di sicurezza}
\end{longtable}
Le modalità di accesso alla rete sono state scelte in modo tale da bilanciare sicurezza ed usabilità, considerando anche le modalità di utilizzo della rete da parte dei relativi fruitori.

\newpage
\subsection{Captive Portal}
\label{table:Risultati attesi per l'implementazione del Captive Portal}
\rowcolors{2}{CRighePari}{CRigheDispari}
\renewcommand*{\arraystretch}{1.2}
\begin{longtable}[H]{p{9.5cm}p{3.4cm}}
	\rowcolor{CHeader}
	\color{CHeaderText} \textbf{Misurazione} & \color{CHeaderText} \textbf{Valore richiesto} \\
	\endhead
	Possibilità di effettuare il login &
	Si \\
	Possibilità di registrazione tramite sponsor &
	Si \\
	Registrazione valida per un determinato periodo di tempo &
	Si \\
	Interfaccia utilizzabile da un dispositivo mobile &
	Si \\
	Registrazione tramite e-mail &
	Si \\
	Possibilità di registrazione con e-mail altrui &
	No \\
	Sezione di amministrazione facilmente accessibile &
	Si \\
	Sezione di amministrazione protetta da password &
	Si \\
	Funzionalità di gestione degli sponsor &
	Si \\
	Funzionalità di gestione degli account &
	Si \\
	Funzionalità di gestione delle durate temporali &
	Si \\
	
	\hiderowcolors
	\caption{Risultati attesi per l'implementazione del Captive Portal}
\end{longtable}
I requisiti scelti sono stati decretati sulla base di soluzioni simili adottate da Wintech S.P.A. e poi adattati alle esigenze del cliente \\

\end{document}
