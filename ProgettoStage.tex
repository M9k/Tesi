\documentclass[Tesi.tex]{subfiles}

\begin{document}
\chapter{Progetto di stage}
Lo scopo dell'attività di stage è l’analisi, la progettazione e l'implementazione di un sistema di monitoraggio della sicurezza fisica di una LAN Campus. \\
Le conoscenze apprese durante lo svolgimento dell'attività consistono nella capacità di progettare ed implementare una soluzione per raggiungere gli obbiettivi prefissati, oltre che ad effettuare attività di tuning per perfezionarla.


\section{Pianificazione}
Le attività sono state suddivise in cinque fasi principali, le quali andranno a coprire tutta la durata del percorso formativo stabilito.

\subsection{Fase 1: Analisi}
\begin{itemize}
	\item \textbf{Periodo}: dal 04/06/2018 al 08/06/2018;
	\item \textbf{Numero di ore}: 40h.
\end{itemize}

L'obiettivo di questa prima fase del percorso è familiarizzare con l'infrastruttura amministrata da Wintech e con i supporti hardware e software da utilizzare per la realizzazione del progetto. \\
Verranno analizzati lo schema di rete dell'infrastruttura, la lista degli apparati e delle loro caratteristiche, il sistema di sicurezza fisica attivi e il sistema di monitoraggio presenti. 

\subsection{Fase 2: Progettazione}
\begin{itemize}
	\item \textbf{Periodo}: dal 11/06/2018 al 22/06/2018;
	\item \textbf{Numero di ore}: 40h.
\end{itemize}
	
Nella seconda fase si procederà alla configurazione del software di monitoraggio Observium e del software di gestione della versione RConfig. \\
Per facilitare le attività verrà definita la nomenclatura da dare ai dispositivi.\\
Ulteriormente si procederà alla definizione delle logiche di sicurezza che verranno implementate basate sul protocollo 802.11X. \\
Durante questa fase verrà anche prodotta la prima bozza della documentazione progettuale.
	

\subsection{Fase 3: Implementazione}
\begin{itemize}
	\item \textbf{Periodo}: dal 25/06/2018 al 06/07/2018;
	\item \textbf{Numero di ore}: 80h.
\end{itemize}
	
In questa fase è stata implementata la nuova infrastruttura e si sono resi operativi i software precedentemente configurati. \\
\`{E} stato creato un laboratorio con uno nuovo switch sul quale sono state testate le politiche precedentemente scelte per valutarne l'efficacia.\\
Una volta completata questa fase si procederà alla creazione delle nuove configurazione per tutti gli switch comprendenti la nuova politica di sicurezza e la loro installazione. \\
Oltre a questo i dispositivi verranno inseriti all'interno del DNS e verranno resi operativi i software Observium e RConfig. \\
Durante questo periodo la documentazione progettuale verrà aggiornata di conseguenza alle attività svolte.\\


\subsection{Fase 4: Test in produzione}
\begin{itemize}
	\item \textbf{Periodo}: dal 09/06/2018 al 13/07/2018;
	\item \textbf{Numero di ore}: 40h.
\end{itemize}
	
La fase di test in produzione è la più importante perché verrà messo alla prova l'intero sistema con il picco dell'utenza. \\
L'attività che verrà svolta è il monitoraggio di eventuali anomalie, per poi procedere con il censirle, cercare la fonte del problema, identificare una soluzione ed implementarla. \\
Durante questo periodo la documentazione progettuale verrà nuovamente aggiornata di conseguenza alle attività svolte.\\

	

\subsection{Fase 5: Tuning}
\begin{itemize}
	\item \textbf{Periodo}: dal 16/07/2018 al 27/07/2018;
	\item \textbf{Numero di ore}: 80h.
\end{itemize}
	
Nella fase finale del progetto sono stati eseguiti tuning su tutta la rete, ove possibili, sia sugli apparati che nelle configurazioni dei software. \\
Questa attività sarà supportata dal software Observium, che nel frattempo avrà raccolto una mole di dati tale da permettere uno studio dei miglioramenti effettuabili.\\
Un'altra attività derivante dallo studio dei dati raccolti sarà la configurazione delle soglie di alarmi automatici, che verranno comunicati tramite messaggio e-mail e Telegram. \\
In questa ultima fase si procederà anche a completare la documentazione. \\

%TODO valutare se ne vale la pena
\begin{comment}
\section{Diagramma di Gantt}
\begin{figure}[H]
	\centering
	\includegraphics[width=1\linewidth]{"images/Gantt"}
	\caption{Diagramma di Gantt della attiviasd svolta}
	\label{fig:Diagramma di Gantt della attiviasd svolta}
\end{figure}
\end{comment}

\section{Obiettivi}
\subsection{Obiettivo aziendale}
L'obiettivo a fine stage è aumentare la sicurezza fisica di una LAN Campus dove accedono migliaia di
persone, impedendo l'uso illecito e non controllato dei servizi di rete, sabotaggi e furto di dati. \\
Questo comprende l'implementazione di software di monitoring e lo sfruttamento dei servizi avanzati che offrono per ottenere le migliori performance ed il miglior controllo possibile.

\subsection{Obiettivo formativo}
L'obiettivo per il tirocinante è acquisire competenze in ambito networking e security in un contesto reale quale una LAN Campus estesa, variegata e con un numero di utenti elevato. \\
Questo al fine di permettergli di mettere in pratica le conoscenze acquisiti durante lo svolgimento dei corsi universitari e fornirgli un forte stimolo ad approfondire ancora di più queste tematiche.

\end{document}
