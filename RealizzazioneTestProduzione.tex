\documentclass[Realizzazione.tex]{subfiles}

\begin{document}
\section{Test finali	}
\begin{itemize}
	\item \textbf{Periodo previsto}: dal 09/07/2018 al 13/07/2018;
	\item \textbf{Numero di ore previste}: 40h;
	\item \textbf{Periodo effettivo}: dal 12/07/2018 al 17/07/2018;
	\item \textbf{Numero di ore effettive}: 32h.
\end{itemize}
I test in produzione non hanno evidenziato gravi problematiche o problemi di progettazione, quindi le ore impiegate in questa attività sono state inferiori a quante preventivate.

\subsection{Individuazione problematiche}
Le problematiche individuate si sono rilevate essere quantitativamente inferiori a quanto previsto. La maggior parte delle modifiche effettuate in questa fase erano mirate soprattutto a migliorare l'usabilità e le interazioni degli utenti, come ad esempio migliorare la responsibilità dell'interfaccia del Captive Portal e aumentare la chiarezza espositiva del testo.\\
Di seguito sono riportate le problematiche affrontate di maggior rilevanza.

\subsubsection{Schedulazione automatica MikroTik non completata}
Si è notato che la schedulazione dei backup di MikroTik non veniva eseguita, ma terminava inaspettatamente allo stabilimento della connessione FTP. \\
Analizzando la problematica si è scoperto che l'esecuzione manuale dello script veniva eseguita ignorando le sue policy, mentre l'esecuzione schedulata le rispettava.\\
Il backup automatico possedeva già il permesso di utilizzare la connessione FTP, ma prima di stabilirla procedeva a controllare se la destinazione era raggiungibile, per tale motivo era richiesta anche la policy "test", che consentiva l'utilizzo del comando ping.

\subsubsection{Adattabilità del Captive Portal a dispositivi mobili}
Lo stile iniziale del Captive Portal prevedeva un footer con delle informazioni aggiuntive, ma lo stile ad esso assegnato era errato in quanto, utilizzando dispositivi con schermo particolarmente piccolo o con i caratteri grandi, andava a coprire il tasto di conferma del login. \\
Per correggere tale problema si è dovuto agire sul file CSS, facendo in modo tale che la parte in basso del sito non andasse mai a sovrapporsi con il contenuto.

\subsection{Aggiornamento della documentazione progettuale}
Durante questa fase è stata aggiornata la documentazione progettuale, andando a descrivere tutti i problemi riscontrati, correlati dalla soluzione applicata per correggerli.

\end{document}
