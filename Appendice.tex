\documentclass[Tesi.tex]{subfiles}

\begin{document}
	
\addcontentsline{toc}{chapter}{Appendici}
\renewcommand{\leftmark}{Appendici}
\chapter*{Appendici}
	
\label{sec:Appendice A}
\addcontentsline{toc}{section}{Appendice A: Schemi di rete}
\section*{Appendice A: Schemi di rete}

\begin{landscape}
\begin{figure}[H]
	\centering
	% NON DI PIù O SALTA ALLA PAGINA SUCCESSIVA!
	\includegraphics[width=0.885\linewidth]{"images/schema_WAN_anonimo_prima"}
	\caption{Schema Visio della rete datato 2015}
	\label{fig:Schema Visio della rete datato 2015}
\end{figure}

\begin{figure}[H]
	\centering
	\includegraphics[width=1\linewidth]{"images/schema_WAN_anonimo_dopo_A"}
	\caption{Schema Visio della rete a fine stage, prima parte}
	\label{fig:Schema Visio della rete a fine stage, prima parte}
\end{figure}

\begin{figure}[H]
	\centering
	\includegraphics[width=1\linewidth]{"images/schema_WAN_anonimo_dopo_B"}
	\caption{Schema Visio della rete a fine stage, seconda parte}
	\label{fig:Schema Visio della rete a fine stage, seconda parte}
\end{figure}
\end{landscape}

\label{sec:Appendice B}
\section*{Appendice B: Script di popolamento rConfig}
\addcontentsline{toc}{section}{Appendice B: Script di popolamento rConfig}
Il codice sotto riportato è uno script in linguaggio JavaScript per il popolamento automatico, una volta inseriti i dati, di rConfig. \\
Lo script può essere utilizzato mediante il plug-in GreaseMonkey versione 4.0 o successive, in quanto richiede l'archiviazione di un valore nella memoria dell'estensione per funzionare. \\


\begin{lstlisting}[caption=Script GreaseMonkey di popolamento rConfig]
// ==UserScript==
// @name        auto-insert-devices-rConfig
// @namespace   MircoCailottoWintech
// @include     https://INDIRIZZO_RCONFIG/devices.php
// @version     1
// @grant       GM.setValue
// @grant       GM.getValue
// ==/UserScript==

// DATA

const data = [
["INDIRIZZO-DISPOSITIVO-IN-DNS", "", "ENABLE-PROMPT#", "MARCA", "MODELLO", "GRUPPO", "LOCAZIONE", "TEMPLATE_DI_RECUPERO_DATI"],
["HP3850-24G-POE-C-ARMADIO-S123", "", "ARMADIO-S123#", "HP", "HP3850-24G-POE", "SwitchesHP", "ClienteA", "HP Procurve SSH no enable - HP-Procurve-SSH-no-enable.yml"],
["HP3850-24G-POE-C-ARMADIO-C23", "", "C-ARMADIO-C23#", "HP", "HP3850-24G-POE", "SwitchesHP", "ClienteA", "HP Procurve SSH no enable - HP-Procurve-SSH-no-enable.yml"],

];

function insertData(index) {
	document.getElementById('deviceName').value = data[index][0];
	document.getElementById('deviceEnablePrompt').value = data[index][1];
	document.getElementById('devicePrompt').value = data[index][2];
	
	var select = document.getElementById('vendorId');
		for (var i = 0; i < select.options.length; i++) {
		if (select.options[i].text === data[index][3]) {
			select.selectedIndex = i;
			break;
		}
	}
	
	document.getElementById('deviceModel').value = data[index][4];
	
	select = document.getElementById('catId');
	for (var i = 0; i < select.options.length; i++) {
		if (select.options[i].text === data[index][5]) {
			select.selectedIndex = i;
			break;
		}
	}

	document.getElementById('custom_Location').value = data[index][6];

	select = document.getElementById('templateId');
	for (var i = 0; i < select.options.length; i++) {
		if (select.options[i].text === data[index][7]) {
			select.selectedIndex = i;
			break;
		}
	}
	
	//LOGIN
	document.getElementById('defaultCreds').checked = true;
	// document.getElementById('deviceUsername').value = "username";
	// document.getElementById('devicePassword').value = "password";
	// document.getElementById('deviceEnablePassword').value = "passwordroot";
};

function clickOkButton() {
	setTimeout(function(){
		unsafeWindow.resolveDevice(document.getElementById('deviceName').value);
		setTimeout(function(){
			document.getElementById("submit").click();
		}, 500);
	}, 500);
};

window.addEventListener('load', function() {
	// once loaded
	(async () => {
		console.log("Async call");
		
		// -----   RESET THE SCRIPT   -----
		var reset = false;
		
		if(reset) {
			GM.setValue('count', 0);
		} else {
			var index = await GM.getValue('count', 0);
			
			if(index < data.length) {
				//insert
				console.log("Adding item number:");
				console.log(index);
				insertData(index);
				GM.setValue('count', index + 1);
				clickOkButton();
			} else {
			//done, do nothing
				console.log("Already done");
			}   
		}
		})();
}, false);

\end{lstlisting}

Le linee 13, 14 e 15 sono 3 apparati che saranno inseriti alla esecuzione dello script, i campi sono:
\begin{enumerate}
	\item Indirizzo DNS del dispositivo
	\item Prompt del dispositivo in modalità non privilegiata, opzionale
	\item Prompt del dispositivo in modalità privilegiata
	\item Marca del dispositivo
	\item Modello del dispositivo
	\item Gruppo rConfig di appartenenza del dispositivo, utilizzabile per il filtraggio
	\item Locazione del dispozitivo, utilizzabile per il filtraggio
	\item Template rConfig relativo alla configurazione da utilizzare per il recupero dei dati
\end{enumerate}

Le linee 54, 55 e 56 presentano la possibilità di specificare i parametri per effettuare il login ai dispositivi, attualmente non utilizzati in quanto si utilizza le credenziali impostate come di default. \\
La linea 74 presenta un variabile "reset" che se impostata a true, invece che eseguire lo script, va ad azzerare i valori utilizzati dal plugin, permettendone una seconda esecuzione.


\label{sec:Appendice C}
\section*{Appendice C: Script di backup Mikrotik}
\addcontentsline{toc}{section}{Appendice C: Script di backup Mikrotik}

Di seguito è riportato lo script eseguito e schedulato sui dispositivi MikroTik, sia di tipologia router che access point. La sua esecuzione consiste nella generazione automatica di un file di backup delle impostazioni e il suo salvataggio su server \gloss{FTP} remoto.

\begin{lstlisting}[caption=Script per il backup dei device MikroTik]
# -----------  Configuration  ------------
:local ftp_server_address backupmikrotik.ftpserver.local
:local ftp_username ftp_username
:local ftp_password ftp_password
# ----------------------------------------	
# Get the date
:local ts [/system clock get time]
:set ts ([:pick $ts 0 2]."-".[:pick $ts 3 5]."-".[:pick $ts 6 8])
:local months {"jan"=1;"feb"=2;"mar"=3;"apr"=4;"may"=5;"jun"=6;"jul"=7;"aug"=8;"sep"=9;"oct"=10;"nov"=11;"dec"=12}
:local ds [/system clock get date]
:local month [:pick $date 0 3]
:local mm (:$months->$month)
:set ds ([:pick $ds 7 11]."$mm-".[:pick $ds 4 6])

# Compose the filename
:local backupFileName "ExportMikrotikSettingsController-$ds-$ts"

# Export the configuration
/export file=$backupFileName

# Add the automatic extension to the filename
:set backupFileName "$backupFileName.rsc"

# Upload the file to the ftp server
/tool fetch mode=ftp address=$ftp_server_address port=21 user=$ftp_username password=$ftp_password src-path=$backupFileName dst-path=$backupFileName upload=yes

# Remove the file from the device
/file remove $backupFileName
\end{lstlisting}

Le prime 3 dichiarazioni di variabili locali sono le informazioni di accesso FTP al server dedito al mantenimento dei backup. \\
Se si desidera si può modificare il nome del file agendo sulla linea 17, utile per avere nomi distinti in presenza di più dispositivi. \\
Agendo invece sulla linea 26 si può appendere un percorso al valore di dst-path, in modo da salvare il file in una subdirectory del server FTP. \\

\end{document}
