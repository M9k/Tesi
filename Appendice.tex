\documentclass[Tesi.tex]{subfiles}

\begin{document}
	
\addcontentsline{toc}{chapter}{Appendici}
\renewcommand{\leftmark}{Appendici}
\chapter*{Appendici}
	
\label{sec:Appendice A}
\addcontentsline{toc}{section}{Appendice A: Schemi di rete}
\section*{Appendice A: Schemi di rete}

\begin{landscape}
\begin{figure}[H]
	\centering
	% NON DI PIù O SALTA ALLA PAGINA SUCCESSIVA!
	\includegraphics[width=0.885\linewidth]{"images/schema_WAN_anonimo_prima"}
	\caption{Schema Visio della rete datato 2015}
	\label{fig:Schema Visio della rete datato 2015}
\end{figure}

\begin{figure}[H]
	\centering
	\includegraphics[width=1\linewidth]{"images/schema_WAN_anonimo_dopo_A"}
	\caption{Schema Visio della rete a fine stage, prima parte}
	\label{fig:Schema Visio della rete a fine stage, prima parte}
\end{figure}

\begin{figure}[H]
	\centering
	\includegraphics[width=1\linewidth]{"images/schema_WAN_anonimo_dopo_B"}
	\caption{Schema Visio della rete a fine stage, seconda parte}
	\label{fig:Schema Visio della rete a fine stage, seconda parte}
\end{figure}
\end{landscape}

\label{sec:Appendice B}
\section*{Appendice B: Script di popolamento rConfig}
\addcontentsline{toc}{section}{Appendice B: Script di popolamento rConfig}
Il codice sotto riportato è uno script in linguaggio JavaScript per il popolamento automatico dei dispositivi su rConfig. \\
Lo script può essere utilizzato mediante il plug-in GreaseMonkey versione 4.0 o successive, in quanto richiede l'archiviazione di un valore nella memoria dell'estensione per funzionare. \\


\begin{lstlisting}[caption=Script GreaseMonkey di popolamento rConfig]
// ==UserScript==
// @name        auto-insert-devices-rConfig
// @namespace   MircoCailottoWintech
// @include     https://INDIRIZZO_RCONFIG/devices.php
// @version     1
// @grant       GM.setValue
// @grant       GM.getValue
// ==/UserScript==

// DATA

const data = [
["INDIRIZZO-DISPOSITIVO-IN-DNS", "", "ENABLE-PROMPT#", "MARCA", "MODELLO", "GRUPPO", "LOCAZIONE", "TEMPLATE_DI_RECUPERO_DATI"],
["HP3850-24G-POE-C-ARMADIO-S123", "", "ARMADIO-S123#", "HP", "HP3850-24G-POE", "SwitchesHP", "ClienteA", "HP Procurve SSH no enable - HP-Procurve-SSH-no-enable.yml"],
["HP3850-24G-POE-C-ARMADIO-C23", "", "C-ARMADIO-C23#", "HP", "HP3850-24G-POE", "SwitchesHP", "ClienteA", "HP Procurve SSH no enable - HP-Procurve-SSH-no-enable.yml"],

];

function insertData(index) {
	document.getElementById('deviceName').value = data[index][0];
	document.getElementById('deviceEnablePrompt').value = data[index][1];
	document.getElementById('devicePrompt').value = data[index][2];
	
	var select = document.getElementById('vendorId');
		for (var i = 0; i < select.options.length; i++) {
		if (select.options[i].text === data[index][3]) {
			select.selectedIndex = i;
			break;
		}
	}
	
	document.getElementById('deviceModel').value = data[index][4];
	
	select = document.getElementById('catId');
	for (var i = 0; i < select.options.length; i++) {
		if (select.options[i].text === data[index][5]) {
			select.selectedIndex = i;
			break;
		}
	}

	document.getElementById('custom_Location').value = data[index][6];

	select = document.getElementById('templateId');
	for (var i = 0; i < select.options.length; i++) {
		if (select.options[i].text === data[index][7]) {
			select.selectedIndex = i;
			break;
		}
	}
	
	//LOGIN
	document.getElementById('defaultCreds').checked = true;
	// document.getElementById('deviceUsername').value = "username";
	// document.getElementById('devicePassword').value = "password";
	// document.getElementById('deviceEnablePassword').value = "passwordroot";
};

function clickOkButton() {
	setTimeout(function(){
		unsafeWindow.resolveDevice(document.getElementById('deviceName').value);
		setTimeout(function(){
			document.getElementById("submit").click();
		}, 500);
	}, 500);
};

window.addEventListener('load', function() {
	// once loaded
	(async () => {
		console.log("Async call");
		
		// -----   RESET THE SCRIPT   -----
		var reset = false;
		
		if(reset) {
			GM.setValue('count', 0);
		} else {
			var index = await GM.getValue('count', 0);
			
			if(index < data.length) {
				//insert
				console.log("Adding item number:");
				console.log(index);
				insertData(index);
				GM.setValue('count', index + 1);
				clickOkButton();
			} else {
			//done, do nothing
				console.log("Already done");
			}   
		}
		})();
}, false);

\end{lstlisting}

Le linee 13, 14 e 15 sono 3 apparati che saranno inseriti alla esecuzione dello script, i campi sono:
\begin{enumerate}
	\item Indirizzo DNS del dispositivo
	\item Prompt del dispositivo in modalità non privilegiata, opzionale
	\item Prompt del dispositivo in modalità privilegiata
	\item Marca del dispositivo
	\item Modello del dispositivo
	\item Gruppo rConfig di appartenenza del dispositivo, utilizzabile per il filtraggio
	\item Locazione del dispositivo, utilizzabile per il filtraggio
	\item Template rConfig relativo alla configurazione da utilizzare per il recupero dei dati
\end{enumerate}

Le linee 54, 55 e 56 presentano i parametri per effettuare il login ai dispositivi nel caso non si desiderasse utilizzare le credenziali di default, disabilitabili alla linea 53. \\
La linea 74 presenta un variabile "reset" che, se impostata a true, non esegue l'inserimento ma va ad azzerare i valori utilizzati dal plugin, permettendone una seconda esecuzione.


\label{sec:Appendice C}
\section*{Appendice C: Script di backup Mikrotik}
\addcontentsline{toc}{section}{Appendice C: Script di backup Mikrotik}

Di seguito è riportato lo script eseguito e schedulato sui dispositivi MikroTik, sia di tipologia router che access point. La sua esecuzione consiste nella generazione automatica di un file di backup delle impostazioni e il suo salvataggio su server \gloss{FTP} remoto.

\begin{lstlisting}[caption=Script per il backup dei device MikroTik]
# -----------  Configuration  ------------
# FTP server address
:local ftp_server_address backupmikrotik.ftpserver.local
# FTP username
:local ftp_username ftp_username
# FTP password
:local ftp_password ftp_password
# remote directory
:local remote_directory /backupMikroTik/Router/
# name of the file, without the date
:local file_name ExportMikrotikSettingsController

# -----------  Create strings  ------------
# Get the date
:local ts [/system clock get time]
:set ts ([:pick $ts 0 2]."-".[:pick $ts 3 5]."-".[:pick $ts 6 8])
:local months {"jan"="01";"feb"="02";"mar"="03";"apr"="04";"may"="05";"jun"="06";"jul"="07";"aug"="08";"sep"="09";"oct"="10";"nov"="11";"dec"="12"};
:local ds [/system clock get date]
:local mm (:$months->[:pick $ds 0 3])
:set ds ([:pick $ds 7 11]."$mm-".[:pick $ds 4 6])

# Add the automatic extension to the filename
:set backupFileName "$backupFileName.rsc"

# Compose the remote path
:set remoteBackupFileName "$remote_directory$backupFileName.rsc"

# Compose the filename
:local backupFileName "$file_name-$ds-$ts"

# -----------  Execute the backup  ------------
# Export the configuration
/export file=$backupFileName

# Upload the file to the ftp server
/tool fetch mode=ftp address=$ftp_server_address port=21 user=$ftp_username password=$ftp_password src-path=$backupFileName dst-path=$remoteBackupFileName upload=yes

# Remove the file from the device
/file remove $backupFileName
\end{lstlisting}

Alla riga 3, 5 e 7 sono dichiarate le informazioni di accesso FTP al server dedito al mantenimento dei backup, mentre nella riga 9 viene indicata la cartella remota sulla quale inserire il file.\\
Alla riga 11 viene definito il nome che assumerà il file di backup, con omesse le informazioni sulla data, che saranno aggiunte automaticamente. \\

Lo script, che si suppone sia stato denominato "script\_backup\_ftp", può essere aggiunto alla schedulazione automatica mediante il seguente comando. \\

\begin{lstlisting}[caption=Comando per la schedulazione del backup]
/system scheduler
add
	name=scheduler_backup_ftp \
	interval=1w \
	start-date=jan/01/2018 \
	start-time=03:00:00 \
	on-event="/system script run script_backup_ftp" \
	policy=ftp,read,write,policy,password,sensitive,test
\end{lstlisting}

In questo modo lo script verrà eseguito ogni lunedì alle 3:00 del mattino, in quanto il 1 gennaio 2018 è un lunedì e la cadenza sarà settimanale.


\end{document}
