\documentclass[Tesi.tex]{subfiles}

\begin{document}
\addcontentsline{toc}{chapter}{Bibliografia}
\renewcommand{\leftmark}{Bibliografia}

\chapter*{Bibliografia}
\section*{Documenti in lingua italiana}
\addcontentsline{toc}{section}{Documenti in lingua italiana}
\begin{itemize}
	\item \textbf{Introduzione a CentOS 7}\\
	\nURI{http://www.html.it/articoli/centos-7-una-distro-enterprise-gratuita}
\end{itemize}

\section*{Documenti in lingua inglese}
\addcontentsline{toc}{section}{Documenti in lingua inglese}
\begin{itemize}
	\item \textbf{Documentazione Observium}\\
	\nURI{http://docs.observium.org}
	\item \textbf{Documentazione Microsoft relativa a Active Directory}\\
	\nURI{https://docs.microsoft.com/en-us/windows-server/identity/identity-and-access}
	\item \textbf{Presentazione MikroTik - WiFi Enterprise con CAPsMAN e Windows NPS}\\
	\nURI{www.youtube.com/watch?v=RXkoAimlcM8}
	\item \textbf{Slide MikroTik - WiFi Enterprise con CAPsMAN e Windows NPS}\\
	\nURI{https://mum.mikrotik.com/presentations/EU18/presentation_5159_1523293520.pdf}
	\item \textbf{Manuale per lo scripting su dispositivi MikroTik}\\
	\nURI{https://wiki.mikrotik.com/wiki/Manual:Scripting}
	\item \textbf{Documentazione FreeRADIUS}\\
	\nURI{https://freeradius.org/documentation}
	\item \textbf{Introduzione e guida a Directory Lister}\\
	\nURI{https://www.directorylister.com}
	
\end{itemize}

\end{document}
