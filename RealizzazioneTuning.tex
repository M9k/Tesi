\documentclass[Realizzazione.tex]{subfiles}

\begin{document}
\section{Tuning}
\begin{itemize}
	\item \textbf{Periodo previsto}: dal 16/07/2018 al 27/07/2018;
	\item \textbf{Numero di ore previste}: 80h;
	\item \textbf{Periodo effettivo}: dal 18/07/2018 al 27/07/2018;
	\item \textbf{Numero di ore effettive}: 64h.
\end{itemize}
Questa fase ha visto come attività principali il potenziamento di Observium, espandendone le funzionalità attraverso un upgrade, configurandole e sfruttandole al meglio ed inserendo un sistema di alert tramite messaggi.

\subsection{Upgrade Observium}
Observium si è subito reso molto utile al monitoraggio della rete, questo ha portato alla decisione di acquistarne la versione professionale. \\
Le principali funzionalità offerte rispetto alla versione gratuita, definita Community, sono le seguenti:
\begin{itemize}
	\item Update e fix costanti e non a cadenza di 6 mesi;
	\item Accesso alla repository SVN;
	\item Accesso alla versione beta;
	\item Ragruppamento dei dispositivi e delle interfacce in base alle loro caratteristiche;
	\item Metriche sulla qualità del servizio;
	\item Ragruppamento delle statistiche;
	\item Indicazione della tipologia degli errori di trasmissione;
	\item Ricerca di un dispositivo tramite IP o MAC address;
	\item Supporto da parte del team di sviluppo.
\end{itemize}

\subsection{Allarmi Observium tramite E-mail e Telegram}
Questa attività è stata eseguita nell'ultima frase in modo tale da consentirmi di aver compreso quali situazioni richiedono un avviso ed aver acquisito una sensibilità numerica riguardante i dati trattati e le soglie da utilizzare. \\\\
Per poter utilizzare un sistema di avvisi automatici tramite messaggi è necessario prima definire le situazioni da considerare pericolose, definite come "alert checks". \\
Ogni alerts check riguarda una classe di dispositivi, interfacce o sensori, come ad esempio le porte, i sensori di temperatura o i device stessi. Oltre al contesto si deve specificare, attraverso un metalinguaggio, i valori di soglia ed è possibile inserire dei filtri, in modo da escludere determinati dispositivi in base alle loro proprietà.

\begin{figure}[H]
	\centering
	\includegraphics[width=1.1\linewidth]{"images/alerts"}
	\caption{Alert checks creati in Observium}
	\label{fig:Alerts creati in Observium}
\end{figure}

\newpage
Successivamente sono stati inseriti dei contatti, sia E-mail che identificativi di conversazioni con un bot di \gloss{Telegram}, e relativamente ad ognuno di essi sono stati abilitati gli alert che interessavano al destinatario.

\begin{figure}[H]
	\centering
	\includegraphics[width=0.5\linewidth]{"images/alerttelegram"}
	\caption{Screenshot degli avvisi nell'applicazione Telegram}
	\label{fig:Screenshot degli avvisi nell'applicazione Telegram}
\end{figure}

\newpage
\subsection{Raggruppamento delle interfacce}
Per velocizzare l'analisi della rete è stato ritenuto utile poter consultare lo stato di più interfacce contemporaneamente, per tale ragione si è andato ad utilizzare la funzionalità di raggruppamento delle porte offerto da Observium.

\begin{figure}[H]
	\centering
	\includegraphics[width=1\linewidth]{"images/trafficoponteradio"}
	\caption{Stato dell'utilizzo del ponte radio}
	\label{fig:Stato dell'utilizzo del ponte radio}
\end{figure}

Un esempio è il gruppo delle interfacce sulle quali comunicano il ponte radio. Questo è utile per poter identificare eventuali problemi nel resto della rete, in quanto la connessione wireless a lungo raggio è stata delegata come supporto alla rete cablata ed eventuale backup, quindi se il traffico totale aumenta notevolmente significa che sono presenti delle problematiche.


\subsection{Raggruppamento dispositivi}
Vista la quantità elevata di dispositivi si è deciso di utilizzare il raggruppamento offerto da Observium per categorizzarli. \\
L'operazione è stata molto veloce, in quanto il software permette di effettuare questa operazione in modo automatico attraverso condizioni e espressioni regolari. Questo consente anche un update automatico dei gruppi in caso di eliminazione o di aggiunta di nuovi dispositivi.

\subsection{Completamento della documentazione}
Prima di terminare l'attività di stage è stata completata la documentazione, in modo tale da aggiornarla per illustrare le ultime attività effettuate ed indicare le operazioni atte alla sua manutenzione.

\end{document}
