\documentclass[Realizzazione.tex]{subfiles}

\begin{document}
\section{Tuning}
\begin{itemize}
	\item \textbf{Periodo previsto}: dal 16/07/2018 al 27/07/2018;
	\item \textbf{Numero di ore previste}: 80h;
	\item \textbf{Periodo effettivo}: dal 18/07/2018 al 27/07/2018;
	\item \textbf{Numero di ore effettive}: 64h.
\end{itemize}
Questa fase ha visto come attività principali il potenziamento di Observium, espandendone le funzionalità attraverso un upgrade, configurandole e sfruttandole al meglio ed inserendo un sistema di alert tramite Telegram.
\subsection{In sistema Observium avrà già acquisito dati da oltre un settimana, verranno quindi configurate tutte le soglie di allarmi con notifica via email e instant message Telegram}

\subsection{Upgrade Observium}
Observium si è subito reso molto utile al monitoraggio della rete, questo ha portato alla decisione di acquistarne la versione professionale. \\
Le principali funzionalità offerte rispetto alla versione gratuita, definita Community, sono le seguenti:
\begin{itemize}
	\item Update e fix costanti e non a cadenza di 6 mesi;
	\item Accesso alla repository SVN;
	\item Accesso alla versione beta;
	\item Ragruppamento dei dispositivi e delle interfacce in base alle loro caratteristiche;
	\item Metriche sulla qualità del servizio;
	\item Ragruppamento delle statistiche;
	\item Indicazione della tipologia degli errori di trasmissione;
	\item Ricerca di un dispositivo tramite IP o MAC address;
	\item Supporto da parte del team di sviluppo.
\end{itemize}

\subsection{Raggruppamento porte}
\subsection{Raggruppamento dispositivi}
%TODO
\subsection{Completamento della documentazione}
	
\end{document}
