\documentclass[Realizzazione.tex]{subfiles}

\begin{document}
\section{Progettazione}

\begin{itemize}
	\item \textbf{Periodo previsto}: dal 11/06/2018 al 22/06/2018;
	\item \textbf{Numero di ore previste}: 40h;
	% TODO completare ore
	\item \textbf{Periodo effettivo}: ;
	\item \textbf{Numero di ore effettive}: .
\end{itemize}
Una volta terminata la fase di analisi della rete si è proceduto a progettare le modifiche che saranno implementate. \\

\subsection{Aggiornamenti configurazioni di rete}
%TODO
\subsection{Nuove politiche di sicurezza fisica}
%TODO
\subsection{Nuove politiche di sicurezza logica basata su 802.1X}
%TODO

\subsection{Progettazione Observium}
Prima di installare e mettere in funzione Observium lo si è provato localmente, andando ad individuare pregi e difetti del programma e comprendendo quindi quale fosse il miglior modo di utilizzarlo.\\
Inizialmente è stato configurato inserendo al suo interno 5 apparati di rete, sui quali si sono testate svariate impostazioni per comprendere quali potessero ritornare utili al monitoraggio della rete attuale.\\\\
Si è notato che la posizione GPS rilevata automaticamente del software era troppo imprecisa, ma la funzionalità era di elevata importanza visto l'estensione della rete. Si è dunque deciso di inserirle prelevandole dal software PRTG Network Monitor, nel quale erano state precedentemente inserite. \\
Un'altra opzione che si è rilevata fondamentale era la disabilitazione delle interfacce non in uso, che altrimenti avrebbero generato degli warning. \\
\begin{figure}[H]
	\centering
	\includegraphics[width=1\linewidth]{"images/Observium_porte"}
	\caption{Interfacce visualizzate all'interno di Observium}
	\label{fig:Interfacce visualizzate all'interno di Observium}
\end{figure}
Ulteriormente il software visualizzava anche la locazione come configurata all'interno dei devices alla loro installazione, ma risultava poco indicativa, quindi si è scelto di sovrascriverla, indicando la zona di appartenenza.

\subsection{Progettazione rConfig}
Analogamente a quanto effettuato con Observium, anche per rConfig si è proceduto ad un suo test per comprenderne le potenzialità e la migliore modalità di utilizzo. \\
Sono stati inseriti anche per esso 6 dispositivi, in modo tale da avere almeno un dispositivo per ogni modello di apparato in utilizzo. \\\\
Si sono notati alcuni problemi, sia di utilizzo che di sicurezza, e di conseguenza si è dovuto fare particolare attenzione nella progettazione, in modo da impedire configurazioni errate ed accessi non autorizzati. \\
Uno dei problemi di sicurezza riscontrati, che si è scoperto essere presente anche nelle altre istanze utilizzate dall'azienda, è una vulnerabilità \gloss{SQL injection} che permetteva di ottenere una lista completa dei report effettuati, compresi quelli rimossi o filtrati per l'account in uso. \\
\begin{figure}[H]
	\centering
	\includegraphics[width=1\linewidth]{"images/rConfig_SQLi"}
	\caption{Attacco SQL injection su rConfig, sulla destra tutte le configurazioni navigabili}
	\label{fig:Attacco SQL injection su rConfig, sulla destra tutte le configurazioni navigabili}
\end{figure}
%TODO: parlare del problema di configurazione di quello al porto?
TODO: parlare del problema di configurazione di quello al porto?

\subsection{Definizione name-convention}
La name-convention scelta per la denominazione degli apparati è la seguente:\\
{ \centering
	\begin{ttfamily}
		\large DEVICE-ZONA[-LOCAZIONE][-NUMERO\_INC]\\
	\end{ttfamily}
}
\medskip
Nella quale i campi presenti indicano: 
\begin{itemize}
	\item  \begin{ttfamily}\textbf{DEVICE}\end{ttfamily}: Il modello del dispositivo installato, ad esempio "HP-2520-8-PoE";
	\item  \begin{ttfamily}\textbf{ZONA}\end{ttfamily}: La zona di appartenenza del dispositivo, ad esempio "ZONA-A", "EDIFICI" o "AMMINISTRAZIONE";
	\item  \begin{ttfamily}\textbf{LOCAZIONE}\end{ttfamily}: La locazione geografica utilizzata per identificare il dispositivo, opzionale nel caso basti la zona per l'identificazione univoca dell'armadio;
	\item  \begin{ttfamily}\textbf{NUMERO}\_INC\end{ttfamily}: Numero incrementale, da incrementare nel caso di più apparati nello stesso armadio di rete.
\end{itemize}

Questa convenzione è stata scelta in accordo con il cliente per permettere una identificazione veloce dell'apparato di rete anche ai manutentori, in linea con le denominazione utilizzata internamente. \\\\
La presenza del modello di dispositivo è contrario alla best-practice da seguire, in quanto la sostituzione di un apparato con un modello successivo richiede la sostituzione della voce all'interno del DNS e di conseguenza la riconfigurazione dei software di monitoraggio. Ciò è stato richiesto in quanto si favorisce l'immediatezza dell'identificazione del dispositivo a discapito della manutenibilità, tenendo in considerazione che gli upgrade non sono frequenti. \\

\subsection{Inserimento nomi DNS nella lista degli apparati}
Dopo aver definito la name-convention da utilizzare ed avere avuto l'approvazione dal cliente si è proseguito applicandola a tutti gli apparati. \\
Sono stati individuati tutti i nomi DNS e sono stati riportati all'interno del documento contenente la lista degli apparati.

\subsection{Lista dei test e risultati previsti}
%TODO
\subsection{Produzione della prima bozza della documentazione progettuale}
Durante questo periodo è stata scritta la prima bozza della documentazione. \\
Si è proceduto ad inserire al suo interno la lista degli apparati, lo schema della rete e tutte le informazioni utilizzare durante queste prime settimane.


\end{document}
