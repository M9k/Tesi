\documentclass[Realizzazione.tex]{subfiles}

\begin{document}
\section{Implementazione}

%TODO
\subsection{Creazione di un laboratorio con un nuovo switch dove verranno testate le configurazioni prima di andare in produzione}
%TODO

\subsection{Inserimento su DNS interno degli hosts} 
Per permettere la implementazione di Observium e rConfig era richiesto l'inserimento degli apparati nel DNS. \\
Quindi si è proceduto a comunicare la lista degli ip dei dispositivi ed i relativi nomi al responsabile, in quanto l'accesso diretto al DNS interno del cliente non era consentito a Wintech.

\subsection{Implementazione Virtual Machine}
Per consentire l'esecuzione continua dei software sono state create due Virtual Machine, o macchine virtuali.\\
Il server utilizzato per tale fine è quello del cliente, utilizzante le tecnologie prodotte da VMware per la virtualizzazione. \\
Le due macchine virtuali sono state entrambe create con Linux, in quanto gratuito e già conosciuto nel contesto aziendale.

\subsection{Configurazione di base software RConfig}
Per la configurazione di rConfig si è proceduto ad automatizzare l'inserimento dei dati, che altrimenti avrebbe consumato una ingente quantità di tempo visto il numero di dispositivi presenti. \\
Come base di riferimento per l'inserimento dei dati si è attinto dalla tabella dei dispositivi 
precedentemente realizzata, che è stata esportata in formato \gloss{CSV}.\\
Si è poi andato a sviluppare uno script, visionabile in \hyperref[sec:Appendice B]{Appendice B}, che mediante il GreaseMonkey consentiva il caricamento automatico dei dati una volta posti in forma corretta. \\
Successivamente si è proceduto a modificare il file CSV operando per mezzo di una \gloss{espressione regolare}, al fine di convertirlo in una matrice in linguaggio JavaScript per il suo utilizzo nello script. \\
Per terminare è stato eseguito lo script, che ha proceduto all'inserimento degli apparati. \\

\subsection{Configurazione di base software monitoraggio Observium}
Analogamente con quanto effettuato con RConfig si è proceduto all'inserimento di tutti i dispositivi e alla loro configurazione in Observium. \\
Questa operazione è stata più complessa rispetto all'altro software, in quanto richiedeva una aggiunta iniziale di ogni dispositivo, seguito dall'attesa della sua identificazioni per poi terminare con l'aggiunta delle impostazioni. \\
Per ogni dispositivo si è dovuto inserire, sempre mediante l'aiuto di GreaseMonkey, la sua locazione e le sue coordinate GPS. Successivamente si è anche dovuto segnalare al software tutte le interfacce non utilizzate, in modo tale che non venisse generato un warning a causa della loro inoperatività.

\subsection{Backup delle configurazioni di tutti gli apparati switch della Lan Campus con RConfig} 
%TODO
asd
\subsection{Predisporre servizio NPS ( Radius ) sui due Active Directory servers} 
%TODO
asd
\subsection{Creazione della configurazione di test per laboratorio con nuove funzionalità di sicurezza} 
%TODO
asd
\subsection{Test nuove funzionalità con switch laboratorio} 
%TODO
asd
\subsection{Creazione nuove configurazioni per tutti gli switch del Campus} 
%TODO
asd
\subsection{Caricamento configurazioni negli switch del Campus} 
%TODO
asd
\subsection{Test di base nuove funzionalità di sicurezza implementate} 
%TODO
asd

\subsection{Schedulazione attività di backup delle impostazioni mediante rConfig}
Una volta verificata che la raccolta della configurazione di ogni singolo dispositivo andasse a buon fine si è proceduto schedulando la sua esecuzione automatica. \\
Questa attività si è rilevata più ostica del previsto, a causa di molte problematiche di rConfig, che non ha ancora raggiunto una versione stabile. \\
Inizialmente si sono notati un numero di backup superiori a quanto schedulato. Questo avveniva perché le attività schedulate e poi rimosse sparivano dall'interfaccia web, ma rimanevano in esecuzione. Per arginare questo problema si è dovuto accedere direttamente al database dell'applicazione mediante MySql e correggere le informazioni in esso contenute. \\
Durante la correzione del problema precedente si sono rilevate altre anomalie all'interno del database, soprattutto dovute a una scarsa normalizzazione dei dati e a una strutturazione non adatta ad un database di tipo SQL. Pertanto si è dovuto controllare eventuali incongruenze e correggerle, in modo da evitare comportamenti inaspettati in futuro.

\subsection{Aggiornamento della documentazione progettuale}
asd

% TODO Observium - METTERE TOTALE DISPOSITIVI E PORTE
	
\end{document}
