\documentclass[Tesi.tex]{subfiles}

\begin{document}
\chapter{Realizzazione}

\section{Analisi}

Una delle prime attività svolte è stata la raccolta della documentazione e la sua analisi. \\
Sono subito emerse enorme incongruenze tra i vari documenti in quanto alcuni erano datati, quindi si è dovuto aggiornarli. \\

\subsection{Lista apparati}
Come base di questa attività si sono utilizzati alcuni documenti Excel con riportate informazioni sparpagliate sui dispositivi, sono stati integrati tra di loro e da questa attività è emerso che c'erano informazioni assenti per alcuni devices, che sono state completate. \\
Successivamente si è andato ad aggiungere la posizione GPS di ogni dispositivo, utilizzando i dati presenti nel software di monitoraggio presente. \\
I dispositivi sono stati nominati secondo la loro locazione in data precedente rispetto l'inizio dello stage. \\

\subsection{Schema di rete}
Per facilitare tutte le attività successive di configurazione e monitoraggio si è proceduto alla redazione di uno schema di rete. \\
Essendo già disponibile uno schema datato 2015 della rete realizzato dal cliente in Microsoft Visio si è scelto di procedere con lo stesso software. \\
A supporto di questo lavoro si sono utilizzate svariate mappe prodotte in tempi e per fini diversi tra di loro, la cui integrazione ha evidenziato degli errori che sono stati segnalati. \\
Lo schema di rete di partenza e quello realizzato, presentanti il nome, il modello di dispositivo e l'indirizzo ip, sono presenti in \hyperref[sec:Appendice A]{Appendice A}.\\
Alcune informazioni riportate sugli schemi sono state redatte o omesse per motivi di sicurezza e di privacy. \\

\subsection{Analisi sistemi di sicurezza fisica}

\subsection{Analisi sistemi di monitoraggio}



\section{Progettazione}

\section{Implementazione}

\section{Test in produzione}

\section{Tuning}

\end{document}
