\documentclass[Tesi.tex]{subfiles}

\begin{document}
\chapter{Tecnologie}

Questa sezione illustra le principali tecnologie utilizzate da Wintech S.P.A per il controllo ed il monitoraggio delle reti di loro competenza. \\
Essendo la sicurezza informatica un settore nel quale gli standard, le leggi e le tipologie di attacco evolvono molto in fretta, le tecnologie utilizzate devono adeguarsi di conseguenza. Questo porta i tool ed i programmi utilizzati a diventare obsoleti oppure variare con il passare del tempo.

\subsection{PRTG Network Monitor}
PRTG è un sistema di monitoraggio della rete, era utilizzato nella rete analizzata durante lo stage prima di Observium per rilevare eventuali anomalie, ma a causa delle limitazioni della versione gratuita e del suo elevato costo si è scelto di sostituirlo.\\
Il software consiste in un servizio al quale ci si collega mediante il client fornito oppure attraverso una interfaccia web e permette di tenere sotto controllo anche siti web e servizi di varia natura. \\
Alla fine dell'attività di stage il software non è stato disabilitato, ma gran parte dei suoi compiti sono ora assolti da Observium, in quanto possiede alcune funzionalità che il suo concorrente non offre. \\
\begin{figure}[H]
	\centering
	\includegraphics[width=0.5\linewidth]{"images/logo/PRTG_logo"}
	\caption{Logo di PRTG}
	\label{fig:Logo di PRTG}
\end{figure}

\subsection{Observium}
Observium è un sistema di monitoraggio della rete compatibile con i dispositivi delle principali aziende produttrici di apparati di rete. \\
Tra i dispositivi supportati si possono citare Cisco, Dell, HP, Huawei, Lenovo, MikroTik, Netgear e ZTE. \\
Il software dispone anche di una ricerca automatica dei dispositivi presente all'interno della rete, utile per reti di piccola-media dimensione. \\
Le funzionalità che fornisce sono il monitoraggio del \gloss{QOS}, il raggruppamento dei dispositivi mediante regole definite dall'utente e l'alert automatico nel caso vengano superate determinate soglie.
\begin{figure}[H]
	\centering
	\includegraphics[width=0.5\linewidth]{"images/logo/Observium_logo"}
	\caption{Logo di Observium}
	\label{fig:Logo di Observium}
\end{figure}

\subsection{RConfig}
Il tool RConfig è un configuration management mirato ai dispositivi di rete che permette in modo veloce ed automatizzato di effettuare una copia delle configurazione degli apparati di rete. \\
\`{E} completamente open source, protetto da licenza GNU v3.0, ed è scritto in linguaggio PHP, questo gli consente di essere installato facilmente su molti sistemi.
\begin{figure}[H]
	\centering
	\includegraphics[width=0.5\linewidth]{"images/logo/rConfig_logo"}
	\caption{Logo di RConfig}
	\label{fig:Logo di RConfig}
\end{figure}

\subsection{GreaseMonkey}
GreaseMonkey è un plugin disponibile per il browser Mozilla Firefox che consente l'esecuzione di script in linguaggio JavaScript. \\
Permette anche lo store di variabili, il caricamento di librerie esterne e l'esecuzione con permessi privilegiati. \\
\`{E} stato utilizzato durante le attività di stage per automatizzare operazioni ripetitive, consentendo di impiegarci meno tempo, da dedicare ad altre attività.
\begin{figure}[H]
	\centering
	\includegraphics[width=0.35\linewidth]{"images/logo/GreaseMonkey_logo"}
	\caption{Logo di GreaseMonkey}
	\label{fig:Logo di GreaseMonkey}
\end{figure}


\subsection{802.1X}
Il 802.1X è uno standard IEEE per il controllo di accesso alla rete port-based, parte della famiglia di protocolli IEEE 802.1. Fornisce un meccanismo di autenticazione mediante una combinazione username/password oppure un certificato digitale.
Questa protezione va ad ampliare la \gloss{sicurezza fisica} delle risorse connesse alla rete, pur essendo di per sè una \gloss{sicurezza logica}, impedendone il raggiungimento ai dispositivi non autorizzati. \\
Lo standard viene implementato all'interno del firmware o del sistema operativo degli apparati di rete e dei dispositivi degli utenti.\\
Funziona sia su connessioni wired che wireless, anche se viene raramente utilizzato nelle connessioni cablate in quanto, ove possibile, viene favorita una protezione fisica perimetrale, apparentemente migliore. \\
Il 802.1X determina unicamente la procedura di autenticazione che deve essere svolta, mentre la definizione degli standard dei messaggi e del trasporto viene definita da altri standard.

\begin{figure}[H]
	\centering
	\includegraphics[width=0.78\linewidth]{"images/8021x-Authentication-message-flow"}
	\caption{Procedura di autenticazione 802.1X}
	\label{fig:Procedura di autenticazione 802.1X}
\end{figure}

\begin{figure}[H]
	\centering
	\includegraphics[width=1\linewidth]{"images/Schema_tecnologie"}
	\caption{Schema delle tecnologie per l'attualizzazione di 802.1X}
	\label{fig:Schema delle tecnologie per l'attualizzazione di 802.1X}
\end{figure}

\subsection{Extensible Authentication Protocol}
Extensible Authentication Protocol, usualmente chiamato semplicemente EAP, è un framework di autenticazione che definisce la struttura dei messaggi. \\
Lo standard include svariate tipologie di messaggio, in modo tale da poter fornire svariate metodologie di login, come ad esempio combinazioni username e password, utilizzo di certificati o mediante un segreto condiviso. \\
Non essendo un protocollo di rete ha la necessità di essere incapsulato in un messaggio per poter essere inviato, ad esempio all'interno del protocollo \gloss{RADIUS} oppure nel protocollo di rete dedicato \gloss{EAPoL}.

\subsection{Windows Network Policy Server}
Network Policy Server, spesso abbreviato in NPS, in italiano "Server dei criteri di rete", è una tecnologia per il controllo dell'accesso alla rete. \\
Decreta chi può accedere alla rete ponendo delle restrizioni ed utilizzare il protocollo \gloss{RADIUS} per comunicare. \\
Un server che lo implementa può decretare indipendentemente l'accesso o meno dei vari dispositivi, costituendo quindi un RADIUS server, oppure può inoltrare la richiesta ad un altro server NPS, quindi coprendo il ruolo di RADIUS proxy.

\subsection{Active Directory}
Active Directory è un insieme di servizi di rete adottati dai sistemi operativi Microsoft. \\
Sono gestiti da un \gloss{Domain Controller} e gestisce le modalità di accesso alle risorse da parte degli utenti. \\
Le risorse possono essere account utente, account relativi a un computer, cartelle condivise, stampanti e servizi di varia natura. I privilegi di accesso alle varie risorse sono determinati mediante dei criteri di gruppo, applicati sugli utenti.

\subsection{CAPsMAN}
CAPsMAN, per esteso Controlled Access Point system Manager, traducibile in "sistema di gestione di access point controllati", è una funzionalità per la gestione degli accessi fornita da MikroTik assieme ai suoi access point. \\
Permette di identificare i client che si connettono comunicando con un server RADIUS, per poi reindirizzarli automaticamente verso una VLAN adeguata al loro ruolo.

%TODO: immagine come in https://i.stack.imgur.com/A5XXk.png per capire come le tecnologie vanno assieme


\begin{comment}
\subsection{QRadar}
Il software QRadar, prodotto da IBM, consente la rilevazione di anomalie all'interno di una rete, sfruttando l'intelligenza artificiale per la rimozione dei falsi positivi e l'individuazione di comportamenti inaspettati.\\
Permette la raccolta, l'aggregazione e l'analisi di file di log da migliaia di dispositivi, in modo da non lasciare scoperto nessun possibile accesso non autorizzato. \\
Questo software risulta essenziale per coloro che si occupano di quest'ambito di sicurezza, in quanto la lettura e l'incrocio dei dati ottenuti dai log di tutti i dispositivi risulta esosa, se non impossibile, per un singolo team di persone.
\begin{figure}[H]
\centering
\includegraphics[width=0.5\linewidth]{"images/logo/QRadar_logo"}
\caption{Logo di QRadar}
\label{fig:Logo di QRadar}
\end{figure}

\subsection{BigFix}
A differenza di tutti gli altri software sopra citati, BigFix non lavora sui dispositivi di rete ma sui dispositivi degli utenti, gestendo il loro stato e soprattutto gli aggiornamenti del sistema e dei software in essi presenti. \\
Il programma quindi rende facile e veloce l'applicazione di patch di sicurezza e il controllo dei permessi degli utenti, in modo tale che eventuali vettori malevoli non possano sfruttare exploit conosciuti presenti nei software utilizzati. \\
Un altro utilizzo è la configurazione automatica di nuovi dispositivi inseriti nella rete, permettendo una installazione automatizzata di nuovi computer, evitando quindi errori di distrazione nella loro configurazione, i quali potrebbero generare problemi di sicurezza.
\begin{figure}[H]
\centering
\includegraphics[width=0.5\linewidth]{"images/logo/BigFix_logo"}
\caption{Logo di BigFix}
\label{fig:Logo di BigFix}
\end{figure}
\end{comment}




\end{document}
