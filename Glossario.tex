\newglossaryentry{System Integrator}{
	name=System Integrator,
	description={Con il termine System Integrator viene indicata una azienda che si occupa di far dialogare impianti diversi tra di loro allo scopo di creare una nuova struttura funzionale che possa utilizzare le potenzialità di impianti d'origine e creare quindi funzionalità originariamente non presenti}
}

\newglossaryentry{Digital Transformation}{
	name=Digital Transformation,
	description={La Digital Transformation, o trasformazione digitale, è quell'insieme di cambiamenti nei comportamenti aziendali e di business collegato e veicolato dalla tecnologia digitale, tramite il quale è possibile traguardare una maggiore competitività di mercato}
}


\newglossaryentry{QOS}{
	name=Quality Of Service,
	description={La qualità del servizio, documentata mediante \gloss{RFC}, è la descrizione o la misurazione delle performance di un servizio di rete, secondo la visione da parte dell'utente a seconda della possibile attività che sta svolgendo}
}


\newglossaryentry{RFC}{
	name=RFC,
	description={Un RFC o Request For Comments, in italiano "richiesta di commenti", è un documento pubblicato dalla Internet Engineering Task Force, che riporta informazioni riguardanti nuove ricerche, innovazioni e metodologie dell'ambito informatico}
}

\newglossaryentry{sicurezza fisica}{
	name={Sicurezza fisica},
	text={sicurezza fisica},
	description={Per sicurezza fisica si intendono il complesso di soluzioni tecnico-pratiche il cui obiettivo è quello di impedire che utenti non autorizzati possano accedere a risorse, sistemi, impianti, dispositivi, apparati, informazioni e dati di natura riservata}
}

\newglossaryentry{sicurezza logica}{
	name={Sicurezza logica},
	text={sicurezza logica},
	description={Per sicurezza logica si intendono il complesso di soluzioni che impediscono ad utenti non autorizzati di compiere azioni che richiedono dei privilegi più elevati rispetto a quelli in loro possesso}
}

\newglossaryentry{SQL injection}{
	name={SQL injection},
	text={SQL injection},
	description={SQL injection è una tecnica di code injection, usata per attaccare applicazioni di gestione dati, con la quale vengono inserite delle stringhe di codice SQL malevole all'interno di campi di input. }
}

\newglossaryentry{Man In The Middle}{
	name={Man In The Middle},
	text={Man In The Middle},
	description={L'attcco Man In The Middle, spesso indicato con "MITM", consiste nella modifica delle tabelle ARP degli apparati di rete al fine di inserirsi in una comunicazione. Questo consente all'attaccante di analizzare il traffico del dispositivo ed eventualmente manometterlo}
}

\newglossaryentry{CSV}{
	name={CSV},
	text={csv},
	description={Il formato CSV, ovvero Comma-Separated Values, è basato su file di testo composti da righe presentanti valori separati da una virgola. Non esiste uno standard formale che lo definisca, ma solamente alcune prassi più o meno consolidate}
}

\newglossaryentry{espressione regolare}{
	name={Espressione regolare},
	text={espressione regolare},
	description={Una espressione regolare, in inglese Regular Expression, Regex o RE, è una sequenza di simboli che identifica un insieme di stringhe. Essa costituisce una funzione che prende in ingresso una stringa e ne restituisce una seconda}
}

\newglossaryentry{Domain Controller}{
	name={Domain Controller},
	text={Domain Controller},
	description={Un Domain Controller (DC) è un server che, nell'ambito di un dominio, attraverso Active Directory (AD), gestisce le richieste di autenticazione per la sicurezza e organizza la struttura del dominio in termini di utenti, gruppi e risorse di rete fornendo dunque un servizio di directory service}
}


\newglossaryentry{RADIUS}{
	name={RADIUS},
	text={RADIUS},
	description={Il protocollo di rete RADIUS fornisce una autenticazione centralizzata ed opera sulla porta 1812. Viene spesso utilizzato con 802.1X. Con il termine RADIUS si può indicare anche il server che fornisce il servizio di autenticazione mediante questo protocollo}
}


\newglossaryentry{EAPoL}{
	name={EAPoL},
	text={EAPoL},
	description={EAP over Lan, abbreviato in EAPoL, è un protocollo di rete generico che permette di incapsulare il protocollo EAP per essere trasmesso}
}

\newglossaryentry{captive portal}{
	name={Captive portal},
	text={captive portal},
	description={Un Captive portal è una pagina web che viene mostrata agli utenti di una rete di telecomunicazioni quando tentano di connettersi ad Internet mediante una richiesta http del loro browser}
}

\newglossaryentry{FTP}{
	name={FTP},
	text={FTP},
	description={Il protocollo FTP, per esteso File Transfer Protocol, permette la trasmissione di dati tra host. Funziona sul protocollo TCP ed utilizza una architettura client-server, anche se è possibile trasferire dati anche tra server}
}



