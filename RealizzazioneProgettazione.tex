\documentclass[Realizzazione.tex]{subfiles}

\begin{document}
\section{Progettazione}

\begin{itemize}
	\item \textbf{Periodo previsto}: dal 11/06/2018 al 22/06/2018;
	\item \textbf{Numero di ore previste}: 80h;
	\item \textbf{Periodo effettivo}: dal 11/06/2018 al 26/06/2018;
	\item \textbf{Numero di ore effettive}: 96h.
\end{itemize}
Una volta terminata la fase di analisi della rete si è proceduto a progettare tutti i software e le configurazioni che saranno implementate. \\
Lo svolgimento di questa attività ha occupato una durata temporale maggiore di quella prevista, in quanto è stato necessario avere un dialogo con il cliente per meglio capire le sue necessità e si è dovuto individuare e studiare le tecnologie da utilizzare.

\subsection{Progettazione Observium}
Prima di installare e mettere in funzione Observium lo si è provato localmente, andando ad individuare pregi e difetti del programma e comprendendo quindi quale fosse il miglior modo di utilizzarlo.\\
Inizialmente è stato istanziato inserendo al suo interno 5 apparati di rete, sui quali si sono testate svariate configurazioni per comprendere gli aspetti che potessero ritornare utili al monitoraggio della rete e quelli che, invece, non erano efficaci per il contesto.\\\\
Si è notato che la posizione GPS rilevata automaticamente del software era troppo imprecisa, ma la funzionalità è di elevata importanza visto l'estensione della rete. Si è dunque deciso di inserirle manualmente, sovrascrivendo il dato presente. \\
Un'altra opzione che si è rilevata fondamentale è la disabilitazione delle interfacce non in uso, che avrebbero generato avvertimenti inutili. \\
\begin{figure}[H]
	\centering
	\includegraphics[width=1\linewidth]{"images/Observium_porte"}
	\caption{Interfacce visualizzate all'interno di Observium}
	\label{fig:Interfacce visualizzate all'interno di Observium}
\end{figure}
Ulteriormente il software visualizzava anche l'area di appartenenza dei devices prelevandola dalla loro configurazione, ma risultava poco indicativa, quindi si è scelto di sovrascriverla indicando la locazione con più precisione.

\subsection{Progettazione rConfig}
Analogamente a quanto effettuato con Observium, anche per rConfig si è eseguito un test per comprenderne le potenzialità e la migliore modalità di utilizzo. \\
Sono stati inseriti anche per esso 6 dispositivi, in modo tale da avere almeno un dispositivo per ogni modello di apparato in utilizzo. \\\\
Si sono notati da subito alcuni problemi, sia di utilizzo che di sicurezza, e di conseguenza si è dovuto fare particolare attenzione nella progettazione, in modo da impedire malfunzionamenti ed accessi non autorizzati. \\
Uno dei problemi di sicurezza riscontrati, che si è scoperto essere presente anche nelle altre istanze utilizzate dall'azienda, è una vulnerabilità \gloss{SQL injection} che permetteva di ottenere una lista completa dei report effettuati, compresi quelli rimossi o non accessibili per l'account in uso. \\
\begin{figure}[H]
	\centering
	\includegraphics[width=1\linewidth]{"images/rConfig_SQLi"}
	\caption{Attacco SQL injection su rConfig, sulla destra tutte le configurazioni navigabili}
	\label{fig:Attacco SQL injection su rConfig, sulla destra tutte le configurazioni navigabili}
\end{figure}

\subsection{Configurazioni degli apparati Access Point}
Per permettere una estensibilità della rete senza l'intervento di un tecnico per la sua configurazione si è scelto di utilizzare CAPsMAN. \\
La struttura della tecnologia CAPsMAN si basa su un router centrale, prodotto da MikroTik, che gestisce e fornisce le configurazioni. Su questo router vengono collegati, mediante la VLAN a lui dedicata, gli Access Point, che provvederanno autonomamente a recuperare ed applicare le impostazioni corrette. \\\\
Questa metodologia permette, in caso di estensione della rete, di evitare una configurazione manuale dei dispositivi, che potrebbe presentare errori di distrazione e compromettere la sicurezza. \\
Un'altra caratteristica di estrema importanza è che le impostazioni, essendo centralizzate, possono essere modificate unicamente nel router e le modifiche vengono poi trasmesse automaticamente a tutti i dispositivi. \\\\
Il sistema così organizzato però introduce un potenziale pericolo, in quanto il router che gestisce CAPsMAN è unico e rappresenta un single point of failure.

\newpage
\subsection{Politiche di sicurezza}
Durante la progettazione si è decretato anche il numero di SSID wireless da impiegare e le loro modalità di accesso. \\
L'attenzione è stata posta sulle reti utilizzate dal personale, in quanto consentono l'accesso a risorse di elevata importanza e per tale ragione devono essere protette. \\
Per tali motivazioni si è scelto di creare 3 reti wireless, così definite:
\begin{itemize}
	\item \textbf{Corporate}: rete dedita alla connessione dei computer dediti ad attività lavorative, consente l'accesso alle risorse;
	\item \textbf{Mobile}: rete dedita alla connessione dei dispositivi personali e/o lavorativi dei dipendenti, come smartphone e tablet, che non richiedono l'accesso a risorse;
	\item \textbf{Guest}: rete dedita a fornire una connettività internet agli ospiti degli uffici.
\end{itemize}

Le misure di autenticazioni sono le seguenti:
\begin{itemize}
	\item \textbf{Corporate}: l'accesso è consentito previa autenticazione mediante EAP utilizzando le proprie credenziali Active Directory e possedendo il relativo certificato digitale;
	\item \textbf{Mobile}: l'accesso è consentito previa autenticazione mediante EAP utilizzando le proprie credenziali Active Directory, senza la necessità di possedere il certificato digitale;
	\item \textbf{Guest}: l'accesso è consentito mediante l'inserimento di una password di bassa complessità e la registrazione su un \gloss{captive portal}.
\end{itemize}

Per l'attualizzazione di tali politiche non sarà sufficiente utilizzare il router MikroTik per determinare l'accettazione o il rifiuto delle richieste di autenticazione, ma saranno necessari dei servizi esterni, che verranno contattati mediante RADIUS. \\\\
Per quanto riguarda la rete Corporate verrà innanzitutto verificato il certificato digitale, per tale azione non ci sarà necessità di appoggiarsi a servizi supplementari, in quanto la loro validità sarà controllata dal router stesso. Per la seconda parte dell'autenticazione, così come per la rete Mobile, si dovrà andare a effettuare un riscontro tra i dati inseriti e gli account presenti nell'Active Directory, per tale motivo ci sarà la necessità di usare un servizio server dei criteri di rete, denominato NPS.
\begin{figure}[H]
	\centering
	\includegraphics[width=1.1\linewidth]{"images/Schema_tecnologie_NPS"}
	\caption{Schema dei dispositivi e dei protocolli usati per le reti Corporate e Mobile}
	\label{fig:Schema dei dispositivi e dei protocolli usati per le reti Corporate e Mobile}
\end{figure}
Infine, per la rete Guest, ci sarà la necessità di avere un servizio RADIUS esterno al router al quale effettuare le richieste, in modo tale che sia sicuro, facilmente raggiungibile e mantenibile mediante una interfaccia web.
\begin{figure}[H]
	\centering
	\includegraphics[width=1.1\linewidth]{"images/Schema_tecnologie_RADIUS"}
	\caption{Schema dei dispositivi e dei protocolli usati per la rete Guest}
	\label{fig:Schema dei dispositivi e dei protocolli usati per la rete Guest}
\end{figure}

\subsection{Captive portal}
Per la progettazione del \gloss{captive portal} si è scelto di utilizzare come modello uno già in utilizzo in alcune soluzioni sviluppate da Wintech, in modo tale da dedicare meno tempo alla progettazione e porre l'attenzione sugli aspetti di networking e di sicurezza. \\\\
La schermata di login si presenta con alcune form dedite all'inserimento della propria e-mail, della selezione di uno sponsor e della durata dell'account. \\
Gli sponsor avranno il compito di confermare o rifiutare la richiesta e potranno essere gestiti dall'amministratore nella apposita area di amministrazione. \\


\begin{figure}[H]
	\centering
	\includegraphics[width=1.1\linewidth]{"images/CaptivePortalReq"}
	\caption{Schermata per la richiesta di un account}
	\label{fig:Schermata per la richiesta di un account}
\end{figure}

Una volta effettuata la richiesta di registrazione verrà inviata una e-mail allo sponsor, ed in caso di approvazione verrà comunicata la password di accesso al richiedente. \\
L'account avrà una durata temporale che non potrà superare un limite massimo deciso dall'amministratore. Questa caratteristica serve per scoraggiare eventuali dipendenti a farne uso per le loro attività lavorative e per impedire un accumularsi di vecchi account registrati ma non più utilizzati. \\\\
La password della rete, di bassa complessità, è stata inserita in modo tale da evitare un possibile spam, o flooding, di e-mail verso gli sponsor. \\

Si è scelto di mantenere la procedura di login all'interno del router MikroTik per evitare redirezioni inutili, il quale andrà a controllare la validità dell'account comunicando con il protocollo RADIUS verso i relativi servizi. \\
La registrazione e la parte di amministrazione invece avverranno in autonomia su una macchina virtuale, che avrà il compito di far interagire l'interfaccia web con il database in modo tale da aggiornare le informazioni contenute nel RADIUS e di avvisare delle richieste e delle conferme riguardanti la registrazione tramite e-mail. \\

Il dispositivo MikroTik presenta un Captive Portal di default, che sarà quindi personalizzato modificandone le grafica ed inserendo tutti i collegamenti verso le pagine di registrazione ed amministrazione presenti sulla macchina virtuale. \\

Di seguito sono riportate alcuni diagrammi per meglio comprendere l'iterazione delle varie parti tra di loro. \\
Per semplificare lo schema è sto omesso l'Autenticator, cioè l'Access Point, in quanto la sua iterazione avviene su un livello inferiore rispetto a quello analizzato attualmente.

\begin{figure}[H]
	\centering
	\includegraphics[width=1.1\linewidth]{"images/CaptivePortal1"}
	\caption{Sequenza per effettuare il login}
	\label{fig:Sequenza per effettuare il login}
\end{figure}
\begin{figure}[H]
	\centering
	\includegraphics[width=1.1\linewidth]{"images/CaptivePortal2"}
	\caption{Sequenza per effettuare la registrazione}
	\label{fig:Sequenza per effettuare la registrazione}
\end{figure}

\subsection{Definizione name-convention}
La name-convention scelta per la denominazione degli apparati è la seguente:\\
{ \centering
	\begin{ttfamily}
		\large DEVICE-ZONA[-LOCAZIONE][-NUMERO\_INC]\\
	\end{ttfamily}
}
\medskip
Nella quale i campi presenti indicano: 
\begin{itemize}
	\item  \begin{ttfamily}\textbf{DEVICE}\end{ttfamily}: Il modello del dispositivo installato, ad esempio "HP-2520-8-PoE";
	\item  \begin{ttfamily}\textbf{ZONA}\end{ttfamily}: La zona di appartenenza del dispositivo, ad esempio "ZONA-A", "EDIFICI" o "AMMINISTRAZIONE";
	\item  \begin{ttfamily}\textbf{LOCAZIONE}\end{ttfamily}: La locazione geografica utilizzata per identificare il dispositivo, opzionale nel caso basti la zona per l'identificazione univoca dell'armadio;
	\item  \begin{ttfamily}\textbf{NUMERO}\_INC\end{ttfamily}: Numero incrementale, da utilizzare nel caso siano presenti più apparati nello stesso armadio di rete.
\end{itemize}

Questa convenzione è stata scelta in accordo con il cliente per permettere una identificazione veloce dell'apparato di rete anche ai manutentori, in linea con le denominazione utilizzata internamente. \\\\
La presenza del modello di dispositivo è contrario alla best-practice da seguire, in quanto la sostituzione di un apparato con un modello successivo richiede la sostituzione della voce all'interno del DNS e di conseguenza la riconfigurazione dei software di monitoraggio. Ciò è stato richiesto in quanto si favorisce l'immediatezza dell'identificazione del dispositivo a discapito della manutenibilità, tenendo in considerazione che gli upgrade non sono frequenti. \\

\subsection{Inserimento nomi DNS nella lista degli apparati}
Dopo aver definito la name-convention da utilizzare ed avere avuto l'approvazione dal cliente si è proseguito applicandola a tutti gli apparati. \\
\'E quindi stato individuato il nome DNS per ogni dispositivo ed è stato riportato all'interno del documento contenente la lista degli apparati.

\subsection{Lista dei test e risultati previsti}
%TODO
TODO - cosa ci scrivo?
\subsection{Produzione della prima bozza della documentazione progettuale}
Durante questo periodo è stata scritta la prima bozza della documentazione. \\
Si è proceduto ad inserire al suo interno la lista degli apparati, lo schema della rete e tutte le informazioni utilizzare durante queste prime settimane.


\end{document}
